\subsection{Proof of Cut Reduction (Lemma~\ref{lemma:cut_reduction})}
\label{subsec:proof_of_cut_reduction_lemma:cut-reduction}
By induction on $d(\Pi_1) + d(\Pi_2)$.  We consider only the case
where the last inferences of $\Pi_1$ and $\Pi_2$ are logical
inferences. The other cases are handled mainly by permutation of
inferences and use of the inductive hypothesis; we refer to Benton's
text for them.  Throughout the proof we will add an asterisk to the
name of an inference rule to indicate that the rule may be applied
zero or more times.

\vspace{1ex}

 \noindent
\emph{J right / J left}. We have 
\begin{center}
\AxiomC{$\pi_1$}
\noLine
\UnaryInfC{$A \vdash_{\mathsf{L}} \Delta, J T^n ; T, \Psi$}
\LeftLabel{$\Pi_1 =$} 
\RightLabel{$\DualLNLLogicdruleLXXjRName$}
\UnaryInfC{$A \vdash_{\mathsf{L}} \Delta, J T^{n+1}; \Psi$}
\DisplayProof
\qquad 
\AxiomC{$\pi_2$}
\noLine
\UnaryInfC{$T \vdash_{\mathsf{C}} \Psi'$}
\LeftLabel{$\Pi_2 =$} 
\RightLabel{$\DualLNLLogicdruleLXXjLName$}
\UnaryInfC{$JT \vdash_{\mathsf{L}} ; \Delta, J T^{n+1}, \Psi'$}
\DisplayProof
\end{center}
By the inductive hypothesis appled to $\Pi_2$ and $\pi_1$ there exists a proof $\Pi'$ of $A \vdash_{\mathsf{L}} \Delta; T, \Psi, \Psi'$
with $c(\Pi') \leq |J T| = |T | + 1$.  Then the following derivation 
\begin{center}
\AxiomC{$\Pi'$}
\noLine
\UnaryInfC{$A \vdash_{\mathsf{L}} \Delta ; T, \Psi, \Psi'$}
\AxiomC{$\pi_2$}
\noLine
\UnaryInfC{$T \vdash_{\mathsf{C}} \Psi'$}
\LeftLabel{$\Pi_1 =$} 
\RightLabel{$\DualLNLLogicdruleLXXCcutName$}
\BinaryInfC{$A \vdash_{\mathsf{L}} \Delta, \Psi, \Psi', \Psi'$}
\doubleLine
\RightLabel{$\DualLNLLogicdruleCXXcrName^*$}
\UnaryInfC{$A \vdash_{\mathsf{L}} \Delta, \Psi, \Psi'$}
\DisplayProof
\end{center}
has cut rank $\mathit{max}( |T|+1, c(\Pi'), c(\pi_2)) = |T|+1 = |J T|$. 

\vspace{1ex}

 \noindent
\emph{H right / H left}. We have 
\begin{center}
\AxiomC{$\pi_1$}
\noLine
\UnaryInfC{$B \vdash_{\mathsf{L}} \Delta, A; H A^n ; \Psi$}
\LeftLabel{$\Pi_1 =$} 
\RightLabel{$\DualLNLLogicdruleLXXhRName$}
\UnaryInfC{$B \vdash_{\mathsf{L}} \Delta; H A^{n+1}, \Psi$}
\DisplayProof
\qquad 
\AxiomC{$\pi_2$}
\noLine
\UnaryInfC{$A \vdash_{\mathsf{L}} ; \Psi'$}
\LeftLabel{$\Pi_2 =$} 
\RightLabel{$\DualLNLLogicdruleCXXhLName$}
\UnaryInfC{$H A \vdash_{\mathsf{C}} ; \Psi'$}
\DisplayProof
\end{center}
By the inductive hypothesis applied to $\Pi_2$ and $\pi_1$ there exists a proof $\Pi'$ of $B \vdash_{\mathsf{L}} \Delta; A, \Psi, \Psi'$
with $c(\Pi') \leq |H A| = |A | + 1$.  Then the following derivation 
\begin{center}
\AxiomC{$\Pi'$}
\noLine
\UnaryInfC{$B \vdash_{\mathsf{L}} \Delta ; A, \Psi, \Psi'$}
\AxiomC{$\pi_2$}
\noLine
\UnaryInfC{$A \vdash_{\mathsf{L}} ; \Psi'$}
\LeftLabel{$\Pi_1 =$} 
\RightLabel{$\DualLNLLogicdruleLXXcutName$}
\BinaryInfC{$B \vdash_{\mathsf{L}} \Delta, \Psi, \Psi', \Psi'$}
\doubleLine
\RightLabel{$\DualLNLLogicdruleCXXcrName^*$}
\UnaryInfC{$B \vdash_{\mathsf{L}} \Delta, \Psi, \Psi'$}
\DisplayProof
\end{center}
has cut rank $\mathit{max}( |A|+1, c(\Pi'), c(\pi_2)) = |A|+1 = |H A|$. 

\vspace{1ex}

 \noindent
\emph{+ right$_1$ / + left}.  We have 
\begin{center}
\AxiomC{$\pi_1$}
\noLine
\UnaryInfC{$S \vdash_{\mathsf{C}} T_1, (T_1 + T_2)^{n}, \Psi$}
\LeftLabel{$\Pi_1 =$} 
\RightLabel{$\DualLNLLogicdruleCXXdROneName$}
\UnaryInfC{$S \vdash_{\mathsf{C}} (T_1 + T_2)^{n+1}, \Psi$}
\DisplayProof
\qquad 
\AxiomC{$\pi_2$}
\noLine
\UnaryInfC{$T_1 \vdash_{\mathsf{C}} \Psi_1$}
\AxiomC{$\pi_3$}
\noLine
\UnaryInfC{$T_2 \vdash_{\mathsf{C}} \Psi_2$}
\LeftLabel{$\Pi_2 =$} 
\RightLabel{$\DualLNLLogicdruleCXXdLName$}
\BinaryInfC{$T_1+ T_2 \vdash_{\mathsf{C}} ; \Psi_1, \Psi_2$}
\DisplayProof
\end{center}

\noindent
If $n = 0$, then the reduction is as follows:
\begin{center}
\begin{tabular}{c}
\AxiomC{$\pi_1$}
\noLine
\UnaryInfC{$S \vdash_{\mathsf{C}} T_1, \Psi$}
\LeftLabel{$\Pi_1 =$} 
\RightLabel{$\DualLNLLogicdruleCXXdROneName$}
\UnaryInfC{$S \vdash_{\mathsf{C}} T_1 + T_2, \Psi$}
\AxiomC{$\pi_2$}
\noLine
\UnaryInfC{$T_1 \vdash_{\mathsf{C}} \Psi_1$}
\AxiomC{$\pi_3$}
\noLine
\UnaryInfC{$T_2 \vdash_{\mathsf{C}} \Psi_2$}
\LeftLabel{$\Pi_2 =$} 
\RightLabel{$\DualLNLLogicdruleCXXdLName$}
\BinaryInfC{$T_1+ T_2 \vdash_{\mathsf{C}}  \Psi_1, \Psi_2$}
\RightLabel{$\DualLNLLogicdruleCXXcutName$}
\BinaryInfC{$S\vdash_{\mathsf{C}} \Psi, \Psi_1, \Psi_2$}
\DisplayProof\\
\\
 reduces to \\ 
\\
\AxiomC{$\pi_1$}
\noLine
\UnaryInfC{$S \vdash_{\mathsf{C}} T_1, \Psi$}
\AxiomC{$\pi_2$}
\noLine
\UnaryInfC{$T_1 \vdash_{\mathsf{C}} \Psi_1$}
\RightLabel{$\DualLNLLogicdruleCXXcutName$}
\BinaryInfC{$S \vdash_{\mathsf{C}} \Psi, \Psi_1$}
\doubleLine
\LeftLabel{$\Pi$}
\RightLabel{$\DualLNLLogicdruleCXXwkName^*$}
\UnaryInfC{$S\vdash_{\mathsf{C}} \Psi, \Psi_1, \Psi_2$}
\DisplayProof
\end{tabular}
\end{center}

\noindent
Here $c(\Pi) = max(|T_1 + 1|, c(\pi_1), c(\pi_2)) \leq |T_1 + T_2|$.

\vspace{1ex}
 
\noindent
If $n > 0$, then by the inductive hypothesis applied to $\Pi_2$ and $\pi_1$ there exists a proof $\Pi'$ of 
$S \vdash_{\mathsf{C}} T_1, \Psi, \Psi_1, \Psi_2$ with $c(\Pi') \leq |T_1+ T_2| = |T_1|+|T_2 | + 1$.  Then the following derivation 
\begin{center}
\AxiomC{$\Pi' $}
\noLine
\UnaryInfC{$S \vdash_{\mathsf{C}} T_1, \Psi, \Psi_1, \Psi_2$}
\AxiomC{$\pi_2$}
\noLine
\UnaryInfC{$T_1 \vdash_{\mathsf{C}}  \Psi_1$}
\LeftLabel{$\Pi =$} 
\RightLabel{$\DualLNLLogicdruleCXXcutName$}
\BinaryInfC{$S \vdash_{\mathsf{C}} \Psi, \Psi_1, \Psi_1, \Psi_2$}
\doubleLine
\RightLabel{$\DualLNLLogicdruleCXXcrName^*$}
\UnaryInfC{$S \vdash_{\mathsf{C}} \Psi, \Psi_1, \Psi_2$}
\DisplayProof
\end{center}
has cut rank $\mathit{max}( |T_1|+1, c(\Pi'), c(\pi_2)) \leq |T_1 + T_2|$. 

\vspace{1ex}

\noindent
$\lsub$ right / $\lsub$ left. We have 
\begin{center}
\begin{tabular}{c}
\AxiomC{$\pi_1$ }
\noLine
\UnaryInfC{$A \vdash_{\mathsf{L}} \Delta_1; \Psi_1, B_1$}
 \AxiomC{$\pi_2$ } 
\noLine
\UnaryInfC{$B_2 \vdash_{\mathsf{L}} \Delta_2; \Psi_2$}
\LeftLabel{$\Pi_1 =$}
\RightLabel{$\DualLNLLogicdruleLXXsRName$}
\BinaryInfC{$A \vdash_{\mathsf{L}} B_1 \lsub B_2,  \Delta_1, \Delta_2; \Psi_1, \Psi_2$}
% 
\AxiomC{$\pi_3$}
\noLine
\UnaryInfC{$B_1 \vdash_{\mathsf{L}} B_2, \Delta ;  \Psi$}
\LeftLabel{$\Pi_2 =$}
\RightLabel{$\DualLNLLogicdruleLXXsLName$}
\UnaryInfC{$B_1 \lsub B_2 \vdash_{\mathsf{L}} \Delta ; \Psi$}
\RightLabel{$\DualLNLLogicdruleLXXcutName$}
\BinaryInfC{$A \vdash_{\mathsf{L}} \Delta_1, \Delta_2, \Delta ; \Psi_1, \Psi_2, \Psi$}
\DisplayProof\\
\\
reduces to $\Pi$ 
\\
\\
\AxiomC{$\pi_1$}
\noLine
\UnaryInfC{$A\vdash_{\mathsf{L}} \Delta_1, B_1; \Psi_1$}
 \AxiomC{$\pi_3$}
\noLine
\UnaryInfC{$B_1 \vdash_{\mathsf{L}} B_2, \Delta ;  \Psi$}
% \LeftLabel{$\Pi'$}
\RightLabel{$\DualLNLLogicdruleLXXcutName$}
\BinaryInfC{$A\vdash_{\mathsf{L}} \Delta_1, \Delta, B_2; \Psi_1,\Psi$}
 \AxiomC{$\pi_2$ } 
\noLine
\UnaryInfC{$B_2 \vdash_{\mathsf{L}} \Delta_2; \Psi_2$}
\RightLabel{$\DualLNLLogicdruleLXXcutName$}
\BinaryInfC{$A \vdash_{\mathsf{L}} \Delta_1, \Delta_2, \Delta; \Psi_1, \Psi_2, \Psi$}
\DisplayProof
\end{tabular}
\end{center}
The resulting derivation $\Pi$ has cut rank $c(\Pi) = max(|B_1|+1, c(\pi_1), c(\pi_2), |B_2|+1, c(\pi_3)) \leq |B_1\lsub B_2|$.  
% section proof_of_cut_reduction_(lemma~\ref{}) (end)

%%% Local Variables: 
%%% mode: latex
%%% TeX-master: main.tex
%%% End: 
