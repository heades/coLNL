\section{Proofs}
\label{sec:proofs}

\subsection{Proof of Lemma~\ref{lemma:symmetric_comonoidal_isomorphisms}}
\label{subsec:proof_of_lemma:symmetric_comonoidal_isomorphisms}
We show that both of the  maps:
\[
\begin{array}{lll}
  \jinv{R,S} := \func{J}R \oplus \func{J}S \mto^{\eta} \func{JH}(\func{J}R \oplus \func{J}S) \mto^{\func{J}\h{A,B}} \func{J}(\func{HJ}R + \func{HJ}S) \mto^{\J(\varepsilon_R + \varepsilon_S)} \func{J}(R + S)\\
  \\
  \jinv{0} := \perp \mto^{\eta} \func{JH}\perp \mto^{\func{J}\h{\perp}} \func{J}0
\end{array}
\]
are mutual inverses with $\j{R,S} : \func{J}(R + S) \mto \func{J}R
\oplus \func{J}S$ and $\j{0} : \perp \mto \func{J}0$ respectively.

\begin{itemize}
\item[Case.] The following diagram implies that $\jinv{R,S};\j{R,S} = \id$:
  \begin{diagram}
    \square|ammm|/->`->`->`<-/<950,500>[
      \func{J}R \oplus \func{J}S`
      \func{JH}(\func{J}R \oplus \func{J}S)`
      \func{JHJ}R \oplus \func{JHJ}S`
      \func{J}(\func{HJ}R + \func{HJ}S);
      \eta`
      \eta \oplus \eta`
      \func{J}\h{}`
      \j{}
    ]
    \dtriangle(-950,0)|amm|/=``<-/<950,500>[
      \func{J}R \oplus \func{J}S`
      \func{J}R \oplus \func{J}S`
      \func{JHJ}R \oplus \func{JHJ}S;``
      \func{J}\varepsilon \oplus \func{J}\varepsilon]

    \qtriangle(-950,-500)/`<-`->/<1900,500>[
      \func{J}R \oplus \func{J}S`
      \func{J}(\func{HJ}R + \func{HJ}S)`
      \func{J}(R + S);`
      \j{}`
      \func{J}(\varepsilon + \varepsilon)]        
  \end{diagram}
  The two top diagrams both commute because $\eta$ and $\varepsilon$
  are the unit and counit of the adjunction respectively, and the
  bottom diagram commutes by naturality of $\j{}$.
  
\item[Case.] The following diagram implies that $\j{R,S};\jinv{R,S} = \id$:
  \begin{diagram}
    \square|ammm|/->`->`->`->/<950,500>[
      \func{J}(R + S)`
      \func{J}R \oplus \func{J}S`
      \func{JHJ}(R + S)`
      \func{JH}(\func{J}R \oplus \func{J}S);
      \j{}`
      \eta`
      \eta`
      \func{JH}\j{}
    ]
    \dtriangle(-950,0)|amm|/=``<-/<950,500>[
      \func{J}(R + S)`
      \func{J}(R + S)`
      \func{JHJ}(R + S);``
      \func{J}\varepsilon]

    \qtriangle(-950,-500)/`<-`->/<1900,500>[
      \func{J}(R + S)`
      \func{JH}(\func{J}R \oplus \func{J}S)`
      \func{J}(\func{HJ}R + \func{HJ}S);`
      \func{J}(\varepsilon + \varepsilon)`
      \func{J}\h{}]
  \end{diagram}
  The top left and bottom diagrams both commute because $\eta$ and $\varepsilon$
  are the unit and counit of the adjunction respectively, and the
  top right diagram commutes by naturality of $\eta$.
  
\item[Case.] The following diagram implies that $\jinv{0};\j{0} = \id$:
  \begin{diagram}
    \square|amma|/->`=`->`<-/<950,500>[
      \perp`
      \func{JH}\perp`
      \perp`
      \func{J}0;
      \eta``
      \func{J}\h{\perp}`
      \j{0}]
  \end{diagram}
  This diagram holds because $\eta$ is the unit of the adjunction.

\item[Case.] The following diagram implies that $\j{0};\jinv{0} = \id$:        
  \begin{diagram}
    \Atriangle|aaa|/->`->`<-/<950,500>[
      \func{JHJ}0`
      \func{J}0`
      \func{JH}\perp;
      \func{J}\varepsilon`
      \func{JH}\j{0}`
      \func{J}\h{\perp}]

    \Dtriangle|aaa|/=`->`/<950,500>[
      \func{J}0`
      \func{JHJ}0`
      \func{J}0;`
      \eta`]

    \square/->``->`/<1900,1000>[
      \func{J}0`
      \perp`
      \func{J}0`
      \func{JH}\perp;
      \j{0}``
      \eta`]
  \end{diagram}
  The top-left and bottom diagrams commute because $\eta$ and
  $\varepsilon$ are the unit and counit of the adjunction
  respectively, and the top-right digram commutes by naturality of
  $\eta$.
\end{itemize}
% subsection proof_of_lemma~\ref{lemma:symmetric_comonoidal_isomorphisms} (end)

\subsection{Proof of Lemma~\ref{lemma:symmetric_comonoidal_monad}}
\label{subsec:proof_of_lemma:symmetric_comonoidal_monad}
Since $\wn$ is the composition of two symmetric comonoidal functors we know it is also symmetric comonoidal, and hence, the following diagrams all hold:
\begin{mathpar}
  \bfig
  \vSquares|ammmmma|/->`->`->``->`->`->/[
    \wn ((A \oplus B) \oplus C)`
    \wn (A \oplus B) \oplus \wn C`
    \wn (A \oplus (B \oplus C))`
    (\wn A \oplus \wn B) \oplus \wn C`
    \wn A \oplus \wn (B \oplus C))`
    \wn A \oplus (\wn B \oplus \wn C);
    \r{A \oplus B,C}`
    \wn \alpha_{A,B,C}`
    \r{A,B} \oplus \id_{\wn C}``
    \r{A,B \oplus C}`
    \alpha_{\wn A,\wn B,\wn C}`
    \id_{\wn A} \oplus \r{B,C}]    
  \efig
\end{mathpar}
%    \and
\begin{mathpar}
  \bfig
  \square|amma|/->`->`->`->/<1000,500>[
    \wn (\perp \oplus A)`
    \wn \perp \oplus \wn A`
    \wn A`
    \perp \oplus \wn A;
    \r{\perp,A}`
    \wn {\lambda}_{A}`
    \r{\perp} \oplus \id_{\wn A}`
      {\lambda^{-1}}_{\wn A}]
  \efig
  \and
  \bfig
  \square|amma|/->`->`->`->/<1000,500>[
    \wn (A \oplus \perp)`
    \wn A \oplus \wn \perp`
    \wn A`
    \wn A \oplus \perp;
    \r{A,\perp}`
    \wn {\rho}_{A}`
    \id_{\wn A} \oplus \r{\perp}`
       {\rho^{-1}}_{\wn A}]
  \efig
\end{mathpar}

\begin{diagram}
  \square|amma|/->`->`->`->/<1000,500>[
    \wn (A \oplus B)`
    \wn A \oplus \wn B`
    \wn (B \oplus A)`
    \wn B \oplus \wn A;
    \r{A,B}`
    \wn {\beta}_{A,B}`
        {\beta}_{\wn A,\wn B}`
        \r{B,A}]
\end{diagram}
Next we show that $(\wn,\eta,\mu)$ defines a monad where
$\eta_A : A \mto ?A$ is the unit of the adjunction, and
$\mu_A = \func{J}\varepsilon_{\func{H}\,A} : \wn\wn A \mto \wn A$.  It
suffices to show that every diagram of
Definition~\ref{def:symm-comonoidal-monad} holds.
\begin{itemize}
\item[Case.]
  $$\bfig
  \square|ammb|<600,600>[
    \wn^3 A`
    \wn^2 A`
    \wn^2 A`
    \wn A;
    \mu_{\wn A}`
    \wn\mu_A`
    \mu_A`
    \mu_A]
  \efig$$
  It suffices to show that the following diagram commutes:
  $$\bfig
  \square|ammb|<600,600>[
    \func{J}(\func{H}(\wn^2 A))`
    \func{J}(\func{H}\,\wn A)`
    \func{J}(\func{H}\,\wn A)`
    \func{J}(\func{H}\,A);
    \func{J}\varepsilon_{\func{H}\,\wn A}`
    \func{J}(\func{H}\,\mu_A)`
    \func{J}\varepsilon_{\func{H}\,A}`
    \func{J}\varepsilon_{\func{H}\,A}]
  \efig$$
  But this diagram is equivalent to the following:
  $$\bfig
  \square|ammb|<600,600>[
    \func{H}\func{JHJH} A`
    \func{H}\,\func{JH} A`
    \func{H}\,\func{JH} A`
    \func{H}\,A;
    \varepsilon_{\func{H}\,\func{JH} A}`
    \func{H}\,\func{J}\varepsilon_{\func{H}\,A}`
    \varepsilon_{\func{H}\,A}`
    \varepsilon_{\func{H}\,A}]
  \efig$$
  The previous diagram commutes by naturality of $\varepsilon$.

\item[Case.]
  $$\bfig
  \Atrianglepair/=`<-`=`->`<-/<600,600>[
    \wn A`
    \wn A`
    \wn^2 A`
    \wn A;`
    \mu_A``
    \eta_{\wn A}`
    \wn \eta_A]
  \efig$$
  It suffices to show that the following diagrams commutes:
  $$\bfig
  \Atrianglepair/=`<-`=`->`<-/<600,600>[
    JH A`
    JH A`
    JHJH A`
    JH A;`
    J\varepsilon_{HA}``
    \eta_{JH A}`
    JH \eta_A]
  \efig$$
  Both of these diagrams commute because $\eta$ and $\varepsilon$
  are the unit and counit of an adjunction.
\end{itemize}

It remains to be shown that $\eta$ and $\mu$ are both
symmetric comonoidal natural transformations, but this easily follows
from the fact that we know $\eta$ is by assumption, and that $\mu$
is because it is defined in terms of $\varepsilon$ which is a
symmetric comonoidal natural transformation.  Thus, all of the
following diagrams commute:
\begin{mathpar}
  \bfig
  \ptriangle|amm|/->`->`<-/<1000,600>[
    A \oplus B`
    \wn A \oplus \wn B`
    \wn (A \oplus B);
    \eta_A \oplus \eta_B`
    \eta_A`
    \r{A,B}]    
  \efig
  \and
  \bfig
  \Vtriangle/->`=`->/<600,600>[
    \perp`
    \wn\perp`
    \perp;
    \eta_\perp``
    \r{\perp}]
  \efig
\end{mathpar}
\begin{mathpar}
  \bfig
  \square|ammm|/->`->``/<1050,600>[
    \wn^2(A \oplus B)`
    \wn (\wn A \oplus \wn B)`
    \wn (A \oplus B)`;
    \wn\r{A,B}`
    \mu_{A \oplus B}``]

  \square(850,0)|ammm|/->``->`/<1050,600>[
    \wn (\wn A \oplus \wn B)`
    \wn^2 A \oplus \wn^2 B``
    \wn A \oplus \wn B;
    \r{\wn A,\wn B}``
    \mu_A \oplus \mu_B`]
  \morphism(-200,0)<2100,0>[\wn(A \oplus B)`\wn A \oplus \wn B;\r{A,B}]
  \efig
  \and
  \bfig
  \square|ammb|/->`->`->`->/<600,600>[
    \wn^2\perp`
    \wn\perp`
    \wn\perp`
    \perp;
    \wn\r{\perp}`
    \mu_\perp`
    \r{\perp}`
    \r{\perp}]
  \efig
\end{mathpar}
% subsection proof_of_lemma:symmetric_monoidal_monad (end)

\subsection{Proof of Lemma~\ref{lemma:right_weakening_and_contraction}}
\label{subsec:proof_of_lemma:right_weakening_and_contraction}
Suppose $(\func{H},\h{})$ and $(\func{J},\j{})$ are two symmetric
comonoidal functors, such that, $\cat{L} : \func{H} \dashv \func{J}
: \cat{C}$ is a dual LNL model.  Again, we know $\wn A = H;J : \cat{L}
\mto \cat{L}$ is a symmetric comonoidal monad by
Lemma~\ref{lemma:symmetric_comonoidal_monad}.  

We define the following morphisms:
\[
\begin{array}{lll}
  \w{A} := \perp \mto^{\jinv{0}} \func{J} 0 \mto^{\func{J}\diamond_{\func{H} A}} \func{JH}A \mto/=/ \wn A\\
  \c{A} := \wn A \oplus \wn A \mto/=/ \func{JH}A \oplus \func{JH}A \mto^{\jinv{\func{H}A,\func{H}A}} \func{J}(\func{H}A + \func{H}A) \mto^{\func{J}\codiag{\func{H}A}} \func{JHA} \mto/=/ \wn A
\end{array}
\]

Next we show that both of these are symmetric comonoidal natural
transformations, but for which functors?  Define $\func{W}(A) =
\perp$ and $\func{C}(A) = \wn A \oplus \wn A$ on objects of
$\cat{L}$, and $\func{W}(f : A \mto B) = \id_\perp$ and $\func{C}(f
: A \mto B) = \wn f \oplus \wn f$ on morphisms.  So we must show
that $\w{} : \func{W} \mto \wn$ and $\c{} : \func{C} \mto \wn$ are
symmetric comonoidal natural transformations.  We first show that
$\w{}$ is and then we show that $\c{}$ is.  Throughout the proof we
drop subscripts on natural transformations for readability.
\begin{itemize}
\item[Case.] To show $\w{}$ is a natural transformation we must show
  the following diagram commutes for any morphism $f : A \mto B$:
  \[
  \bfig
  \square[W(A)`\wn A`W(B)`\wn B;\w{A}`W(f)`\wn f`\w{B}]
  \efig
  \]
  This diagram is equivalent to the following:
  \[
  \bfig
  \square[\perp`\wn A`\perp`\wn B;\w{A}`\id_{\perp}`\wn f`\w{B}]
  \efig
  \]
  It further expands to the following:
  \[
  \bfig
  \hSquares/->`->`->``->`->`->/[\perp`\func{J}0`\func{JH}A`\perp`\func{J}0`\func{JH}B;\jinv{0}`\func{J}(\diamond_{\func{H}A})`\id_\perp``\func{JH}f`\jinv{0}`\func{J}(\diamond_{\func{H}B})]
  \efig
  \]
  This diagram commutes, because
  $\func{J}(\diamond_{\func{H}A});\func{J}f =
  \func{J}(\diamond_{\func{H}A};f) =
  \func{J}(\diamond_{\func{H}B})$, by the uniqueness of the initial
  map.

\item[Case.] The functor $\func{W}$ is comonoidal itself.  To see this we
  must exhibit a map
  \[\s{\perp} := \id_\perp : \func{W}\perp \mto \perp\]
  and a natural transformation
  \[\s{A,B} := \rho^{-1}_\perp : \func{W}(A \oplus B) \mto \func{W}A \oplus \func{W}B\]
  subject to the coherence conditions in
  Definition~\ref{def:coSMCFUN}.  Clearly, the second map is a natural
  transformation, but we leave showing they respect the coherence
  conditions to the reader.  Now we can show that $\w{}$ is indeed
  symmetric comonoidal.
  \begin{itemize}
  \item[Case.] \ \\
    \begin{diagram}
      \square|amma|<1000,500>[
        \func{W}(A \oplus B)`
        \func{W}A \oplus \func{W}B`
        \wn (A \oplus B)`
        \wn A \oplus \wn B;
        \s{A,B}`
        \w{A \oplus B}`
        \w{A} \oplus \w{B}`
        \r{A,B}]
    \end{diagram}
    Expanding the objects of the previous diagram results in the
    following:
    \begin{diagram}
      \square|amma|<1000,500>[
        \perp`
        \perp \oplus \perp`
        \wn (A \oplus B)`
        \wn A \oplus \wn B;
        \s{A,B}`
        \w{A \oplus B}`
        \w{A} \oplus \w{B}`
        \r{A,B}]
    \end{diagram}
    This diagram commutes, because the following fully expanded
    diagram commutes:
    \begin{diagram}
      \square|amma|/<-`->`->`->/<950,500>[
        \J 0`
        \J (0 + 0)`
        \J\H (A \oplus B)`
        \J (\H A + \H B);
        \J\rho`
        \J\diamond`
        \J (\diamond + \diamond)`
        \J\h{}]

      \square(950, 0)|amma|/->``->`->/<950,500>[
        \J (0 + 0)`
        \J 0 + \J 0`
        \J (\H A + \H B)`
        \J\H A \oplus \J\H B;
        \j{}``
        \J\diamond \oplus \J\diamond`
        \j{}]

      \square(0,500)/->`->``/<1900,1500>[
        \perp`
        \perp \oplus \perp`
        \J 0`;
        \rho^{-1}`
        \jinv{0}``]

      \dtriangle(950,1300)|mma|<950,700>[
        \perp \oplus \perp`
        \J 0 \oplus \perp`
        \J 0 \oplus \J 0;
        \jinv{0} \oplus \id`
        \jinv{0} \oplus \jinv{0}`
        \id \oplus \jinv{0}]

      \ptriangle(950,800)|amm|/`=`->/<950,500>[
        \J 0 \oplus \perp`
        \J 0 \oplus \J 0`
        \J 0 \oplus \perp;``
        \id \oplus \j{0}]

      \morphism(1900,1300)|m|/=/<0,-800>[
        \J 0 \oplus \J 0`
        \J 0 \oplus \J 0;]

      \morphism(0,500)|m|<950,800>[
        \J 0`
        \J0 \oplus \perp;
        \rho^{-1}]

      \place(475,250)[(1)]
      \place(1425,250)[(2)]
      \place(950,650)[(3)]
      \place(1180,1100)[(4)]
      \place(1620,1550)[(5)]
      \place(475,1550)[(6)]
    \end{diagram}
    Diagram 1 commutes because $0$ is the initial object, diagram 2
    commutes by naturality of $\j{}$, diagram 3 commutes because
    $\J$ is a symmetric comonoidal functor, diagram 4 commutes
    because $\j{0}$ is an isomorphism
    (Lemma~\ref{lemma:symmetric_comonoidal_isomorphisms}), diagram 5
    commutes by functorality of $\J$, and diagram 6 commutes by
    naturality of $\rho$.
    
  \item[Case.]\ \\
    \begin{diagram}
      \Vtriangle/<-`<-`<-/[
        \perp`
        \wn \perp`
        \func{W}\perp;
        \r{\perp}`
        \s{\perp}`
        \w{\perp}]
    \end{diagram}
    Expanding the objects in the previous diagram results in the
    following:
    \begin{diagram}
      \Vtriangle/<-`=`<-/[
        \perp`
        \wn \perp`
        \perp;
        \r{\perp}``
        \w{\perp}]
    \end{diagram}
    This diagram commutes because the following one does:
    \begin{diagram}
      \dtriangle|ama|/=`<-`->/<950,500>[
        \J 0`
        \J 0`
        \J\H \perp;`
        \J\h{\perp}`
        \J\diamond]
      \square(-950,0)|aaaa|/`=``->/<950,500>[
        \perp``
        \perp`
        \J 0;```
        \jinv{0}]
      \morphism(-950,500)/<-/<1900,0>[
        \perp`
        \J 0;
        \j{0}]        
    \end{diagram}
    The diagram on the left commutes because $\j{0}$ is an
    isomorphism
    (Lemma~\ref{lemma:symmetric_comonoidal_isomorphisms}), and the
    diagram on the right commutes because $0$ is the initial object.

  \end{itemize}

\item[Case.] Now we show that $\c{A} : \wn A \oplus \wn A \mto \wn
  A$ is a natural transformation.  This requires the following
  diagram to commute (for any $f : A \mto B$):
  \[
  \bfig
  \square[
    \func{C}A`
    \wn A`
    \func{C}B`
    \wn B;
    \c{A}`
    \func{C}f`
    \wn f`
    \c{B}]
  \efig
  \]
  This expands to the following diagram:
  \[
  \bfig
  \square[
    \wn A \oplus \wn A`
    \wn A`
    \wn B \oplus \wn B`
    \wn B;
    \c{A}`
    \wn f \oplus \wn f`
    \wn f`
    \c{B}]
  \efig
  \]
  This diagram commutes because the following diagram does:
  \begin{diagram}
    \hSquares[
      \func{JH}A \oplus \func{JH}A`
      \func{J}(\func{H}A + \func{H}A)`
      \func{JH}A`
      \func{JH}B \oplus \func{JH}B`
      \func{J}(\func{H}B + \func{H}B)`
      \func{JH}B;
      \jinv{\func{H}A, \func{H}A}`
      \func{J}\bigtriangledown_{\func{H}A}`
      \func{JH}f \oplus \func{JH}f`
      \func{J}(\func{H}f + \func{H}f)`
      \func{JH}f`
      \jinv{\func{H}B, \func{H}B}`
      \func{J}\bigtriangledown_{\func{H}B}]
  \end{diagram}
  The left square commutes by naturality of $\jinv{}$, and the right square commutes by naturality of the codiagonal
  $\bigtriangledown_{A} : A + A \mto A$.

\item[Case.] The functor $\func{C} : \cat{L} \mto \cat{L}$ is indeed
  symmetric comonoidal where the required maps are defined as follows:
  \[
  \small
  \begin{array}{lll}
    \quad\quad\quad\t_\perp := \wn \perp \oplus \wn \perp \mto/=/ \J\H\perp \oplus \J\H\perp \mto^{\jinv{}} \J(\H\perp + \H\perp) \mto^{\J\codiag{}} \J\H\perp \mto^{\J\h{\perp}} \J 0 \mto^{\j{0}} \perp\\
    \\
    \quad\quad\quad\t_{A,B} := \wn (A \oplus B) \oplus \wn (A \oplus B) \mto^{\r{A,B} \oplus \r{A,B}} (\wn A \oplus \wn B) \oplus (\wn A \oplus \wn B) \mto^{\iso} (\wn A \oplus \wn A) \oplus (\wn B \oplus \wn B)
  \end{array}
  \]
  where $\mathsf{iso}$ is a natural isomorphism that can easily be
  defined using the symmetric monoidal structure of
  $\cat{L}$. Clearly, $\t$ is indeed a natural transformation, but
  we leave checking that the required diagrams in
  Definition~\ref{def:coSMCFUN} commute to the reader.  We can now
  show that $\c{A} : \wn A \oplus \wn A \mto \wn A$ is symmetric
  comonoidal.  The following diagrams from
  Definition~\ref{def:coSMCNAT} must commute:
  \begin{itemize}
  \item[Case.] \ \\
    \begin{diagram}
      \square<1000,500>[
        \func{C}(A \oplus B)`
        \func{C}A \oplus \func{C}B`
        \wn (A \oplus B)`
        \wn A \oplus \wn B;
        \t_{A,B}`
        \c{A \oplus B}`
        \c{A} \oplus \c{B}`
        \r{A,B}]
    \end{diagram}
    Expanding the objects in the previous diagram results in the following:
    \begin{diagram}
      \square<2000,500>[
        \wn (A \oplus B) \oplus \wn (A \oplus B)`
        (\wn A \oplus \wn A) \oplus (\wn B \oplus \wn B)`
        \wn (A \oplus B)`
        \wn A \oplus \wn B;
        \t_{A,B}`
        \c{A \oplus B}`
        \c{A} \oplus \c{B}`
        \r{A,B}]
    \end{diagram}
    This diagram commutes, because the following fully expanded one
    does:
    \begin{center}
      \rotatebox{90}{$
        \bfig
        \square|amma|<1500,500>[
          \J(\H (A \oplus B) + \H (A \oplus B))`
          \J((\H A + \H B) + (\H A + \H B))`
          \J\H (A \oplus B)`
          \J (\H A + \H B);
          \J(\h{} + \h{})`
          \J\codiag{}`
          \J\codiag{}`
          \J\h{}]

        \square(0,500)|amma|<1500,500>[
          \J\H(A \oplus B) \oplus \J\H(A \oplus B)`
          \J(\H A + \H B) \oplus \J(\H A + \H B)`
          \J(\H(A \oplus B) + \H(A \oplus B))`
          \J((\H A + \H B) + (\H A + \H B));
          \J\h{} \oplus \J\h{}`
          \jinv{}`
          \jinv{}`
          \J(\h{} + \h{})]

        \square(1500,0)|amma|/->`->`->`=/<1500,500>[
          \J((\H A + \H B) + (\H A + \H B))`
          \J((\H A + \H A) + (\H B + \H B))`
          \J (\H A + \H B)`
          \J (\H A + \H B);
          \J\iso`
          \J\codiag{}`
          \J(\codiag{} + \codiag{})`]

        \square(3000,0)|amma|<1500,500>[
          \J((\H A + \H A) + (\H B + \H B))`
          \J(\H A + \H A) \oplus \J(\H B + \H B)`
          \J (\H A + \H B)`
          \J\H A \oplus \J\H B;
          \j{}`
          \J(\codiag{} + \codiag{})`
          \J\codiag{} \oplus \J\codiag{}`
          \j{}]

        \dtriangle(3000,500)|ama|/<-`->`->/<1500,500>[
          (\J\H A \oplus \J\H A) \oplus (\J\H B \oplus \J\H B)`
          \J((\H A + \H A) + (\H B + \H B))`
          \J(\H A + \H A) \oplus \J(\H B + \H B);
          \j{};(\j{} \oplus \j{})`
          \jinv{} \oplus \jinv{}`
          \j{}]          

        \morphism(1500,1000)<1500,0>[
          \J(\H A + \H B) \oplus \J(\H A + \H B)`
          (\J\H A \oplus \J\H B) \oplus (\J\H A \oplus \J\H B);
          \j{} \oplus \j{}]

        \morphism(3000,1000)<1500,0>[
          (\J\H A \oplus \J\H B) \oplus (\J\H A \oplus \J\H B)`
          (\J\H A \oplus \J\H A) \oplus (\J\H B \oplus \J\H B);
          \iso]

        \place(750,250)[(1)]
        \place(750,750)[(2)]
        \place(2250,250)[(3)]
        \place(2550,750)[(4)]
        \place(3750,250)[(5)]
        \place(4125,700)[(6)]
        \efig
        $}
    \end{center}
    Diagram 1 commutes by naturality of $\codiag{}$, diagram 2
    commutes by naturality of $\jinv{}$, diagram 3 commutes by
    straightforward reasoning on coproducts, diagram 4 commutes by
    straightforward reasoning on the symmetric monoidal structure of
    $\J$ after expanding the definition of the two isomorphisms --
    here $\J\iso$ is the corresponding isomorphisms on coproducts --
    diagram 5 commutes by naturality of $\j{}$, and diagram 6
    commutes because $\j{}$ is an isomorphism
    (Lemma~\ref{lemma:symmetric_comonoidal_isomorphisms}).
    
  \item[Case.] \ \\
    \begin{diagram}
      \Vtriangle/<-`<-`<-/[
        \perp`
        \wn \perp`
        \func{C} \perp;
        \r{\perp}`
        \t{\perp}`
        \c{\perp}]
    \end{diagram}
    Expanding the objects of this diagram results in the following:
    \begin{diagram}
      \square/<-`<-`<-`=/<950,500>[
        \perp`
        \wn \perp`
        \wn \perp \oplus \wn \perp`
        \wn \perp \oplus \wn \perp;
        \r{\perp}`
        \t{\perp}`
        \c{\perp}`]          
    \end{diagram}
    Simply unfolding the morphisms in the previous diagram reveals the following:
    \begin{diagram}
      \square/`<-`<-`=/<950,500>[
        \J(\H\perp + \H\perp)`
        \J(\H\perp + \H\perp)`
        \J\H\perp \oplus \J\H\perp`
        \J\H\perp \oplus \J\H\perp;`
        \jinv{}`
        \jinv{}`]

      \square(0,500)/`<-`<-`/<950,500>[
        \J\H\perp`
        \J\H\perp`
        \J(\H\perp + \H\perp)`
        \J(\H\perp + \H\perp);
        `
        \J\codiag{}`
        \J\codiag{}`]

      \square(0,1000)/`<-`<-`/<950,500>[
        \J0`
        \J0`
        \J\H\perp`
        \J\H\perp;
        `
        \J\h{\perp}`
        \J\h{\perp}`]

      \square(0,1500)/=`<-`<-`/<950,500>[
        \perp`
        \perp`
        \J0`
        \J0;
        `
        \j{}`
        \j{}`]
    \end{diagram}
    Clearly, this diagram commutes.
  \end{itemize}
\end{itemize}
At this point we have shown that $\w{A} : \perp \mto \wn A$ and
$\c{A} : \wn A \oplus \wn A \mto \wn A$ are symmetric comonoidal
naturality transformations.  Now we show that for any $\wn A$ the
triple $(\wn A,\w{A},\c{A})$ forms a commutative monoid.  This means
that the following diagrams must commute:
\begin{itemize}
\item[Case.]\ \\
  \begin{diagram}
    \hSquares|aammmmm|/->`->```->``/[
      (\wn A \oplus \wn A) \oplus \wn A`
      \wn A \oplus (\wn A \oplus \wn A)`
      \wn A \oplus \wn A```
      \wn A;
      \alpha_{\wn A,\wn A,\wn A}`
      \id_{\wn A} \oplus \c{A}```
      \c{A}``]
    \btriangle|maa|/->``->/<2407,500>[(\wn A \oplus \wn A) \oplus \wn A`
      \wn A \oplus \wn A`
      \wn A;
      \c{A} \oplus \id_{A}``
      \c{A}]
  \end{diagram}
  The previous diagram commutes, because the following one does (we
  omit subscripts for readability):
  \begin{diagram}
    \scriptsize
    \square|amma|/->`->``->/<950,500>[
      (\func{JH}A \oplus \func{JH}A) \oplus \func{JH}A`
      \func{JH}A \oplus (\func{JH}A \oplus \func{JH}A)`
      \func{J}(\func{H}A + \func{H}A) \oplus \func{JH}A`
      \func{J}((\func{H}A + \func{H}A) + \func{H}A);
      \alpha`
      \jinv{} \oplus \id``
      \jinv{}]

    \square(950,0)|amma|/->``->`->/<950,500>[
      \func{JH}A \oplus (\func{JH}A \oplus \func{JH}A)`
      \func{JH}A \oplus \func{J}(\func{H}A + \func{H}A)`
      \func{J}((\func{H}A + \func{H}A) + \func{H}A)`
      \func{J}(\func{H}A + (\func{H}A + \func{H}A));
      \id \oplus \jinv{}``
      \jinv{}`
      \func{J}\alpha]

    \square(1900,0)|amma|/->``->`->/<950,500>[
      \func{JH}A \oplus \func{J}(\func{H}A + \func{H}A)`
      \func{JH}A \oplus \func{JH}A`
      \func{J}(\func{H}A + (\func{H}A + \func{H}A))`
      \func{J}(\func{H}A + \func{H}A);
      \id \oplus \func{J}\codiag{}``
      \jinv{}`
      \func{J}(\id + \codiag{})]

    \square(0,-500)|amma|/`->`->`->/<950,500>[
      \func{J}(\func{H}A + \func{H}A) \oplus \func{JH}A`
      \func{J}((\func{H}A + \func{H}A) + \func{H}A)`
      \func{JH}A \oplus \func{JH}A`
      \func{J}(\func{H}A + \func{H}A);`
      \func{J}\codiag{} \oplus \id`
      \func{J}(\codiag{} + \id)`
      \jinv{}]

    \dtriangle(950,-500)|ama|/`->`->/<1900,500>[
      \func{J}(\func{H}A + \func{H}A)`
      \func{J}(\func{H}A + \func{H}A)`
      \func{JH}A;`
      \func{J}\codiag{}`
      \func{J}\codiag{}]

    \place(950,250)[(1)]
    \place(2375,250)[(2)]
    \place(475,-250)[(3)]
    \place(1900,-250)[(4)]
  \end{diagram}
  Diagram 1 commutes because $\func{J}$ is a symmetric monoidal
  functor (Corollary~\ref{corollary:J-SMMF}), diagrams 2 and 3
  commute by naturality of $\jinv{}$, and diagram 4 commutes because
  $(\func{H}A, \diamond, \codiag{})$ is a commutative monoid in
  $\cat{C}$, but we leave the proof of this to the reader.

\item[Case.]\ \\
  \begin{diagram}
    \btriangle|maa|/->`->`->/<1000,600>[
      \wn A \oplus \perp`
      \wn A \oplus \wn A`
      \wn A;
      \id_{\wn A} \oplus \w{A}`
      \rho_{\wn A}`
      \c{A}]
  \end{diagram}
  The previous diagram commutes, because the following one does:
  \begin{diagram}
    \square|amma|/->`->`->`->/<950,500>[
      \func{JH}A \oplus \func{J}0`
      \func{J}(\func{H}A + 0)`
      \func{JH}A \oplus \func{JH}A`
      \func{J}(\func{H}A + \func{H}A);
      \jinv{}`
      \id \oplus \func{J}\diamond`
      \func{J}(\id \oplus \diamond)`
      \jinv{}]

    \square(950,0)|amma|/->``=`->/<950,500>[
      \func{J}(\func{H}A + 0)`
      \func{JH}A`
      \func{J}(\func{H}A + \func{H}A)`
      \func{JH}A;
      \func{J}\rho```
      \func{J}\codiag{}]

    \square(0,500)|amma|/->`->`=`/<1900,500>[
      \func{JH}A \oplus \perp`
      \func{JH}A`
      \func{JH}A \oplus \func{J}0`
      \func{JH}A;
      \rho`
      \id \oplus \jinv{0}``]

    \place(950,750)[(1)]
    \place(475,250)[(2)]
    \place(1425,250)[(3)]      
  \end{diagram}
  Diagram 1 commutes because $\func{J}$ is a symmetric monoidal
  functor (Corollary~\ref{corollary:J-SMMF}), diagram 2 commutes by
  naturality of $\jinv{}$, and diagram 3 commutes because
  $(\func{H}A, \diamond, \codiag{})$ is a commutative monoid in
  $\cat{C}$, but we leave the proof of this to the reader.
  
\item[Case.]\ \\
  \begin{diagram}
    \btriangle|maa|/->`->`->/<1000,600>[
      \wn A \oplus \wn A`
      \wn A \oplus \wn A`
      \wn A;
      \beta_{\wn A,\wn A}`
      \c{A}`
      \c{A}]
  \end{diagram}
  This diagram commutes, because the following one does:
  \begin{diagram}
    \hSquares/->`->`->`->`=`->`->/[
      \func{JH}A \oplus \func{JH}A`
      \func{J}(\func{H}A + \func{H}A)`
      \func{JH}A`
      \func{JH}A \oplus \func{JH}A`
      \func{J}(\func{H}A + \func{H}A)`
      \func{JH}A;
      \jinv{}`
      \func{J}\codiag{}`
      \beta`
      \func{J}\beta``
      \jinv{}`
      \func{J}\codiag{}]
  \end{diagram}
  The left diagram commutes by naturality of $\jinv{}$, and the right
  diagram commutes because $(\func{H}A, \diamond, \codiag{})$ is a
  commutative monoid in $\cat{C}$, but we leave the proof of this to
  the reader.
\end{itemize}

Finally, we must show that $\w{A} : \perp \mto \wn A$ and $\c{A} :
\wn A \oplus \wn A \mto \wn A$ are $\wn\text{-algebra}$ morphisms.
The algebras in play here are $(\wn A,\mu : \wn\wn A \mto \wn A)$,
$(\perp, \r{\perp} : \wn \perp \mto \perp)$, and $(\wn A \oplus \wn
A, u_A : \wn (\wn A \oplus \wn A) \mto \wn A \oplus \wn A)$, where
$u_A := \wn (\wn A \oplus \wn A) \mto^{\r{\wn A,\wn A}} ?^2 A \oplus
?^2 A \mto^{\mu_A \oplus \mu_A} \wn A \oplus \wn A$.  It suffices to
show that the following diagrams commute:
\begin{itemize}
\item[Case.]\ \\
  \begin{diagram}
    \square<950,500>[
      \wn \perp`
      \perp`
      \wn\wn A`
      \wn A;
      \r{\perp}`
      \wn\w{}`
      \w{}`
      \mu]
  \end{diagram}
  This diagram commutes, because the following fully expanded one does:
  \begin{diagram}
    \square|mmma|<2500,500>[
      \J\H\J 0`
      \J 0`
      \J\H\J\H A`
      \J\H A;
      \J\varepsilon_0`
      \J\H\J\diamond`
      \J\diamond`
      \J\varepsilon]

    \square(0,500)|amma|/->`->``/<1250,1000>[
      \J\H\perp`
      \J 0`
      \J\H\J 0`;
      \J\h{\perp}`
      \J\H\jinv{0}``]
    \square(1250,500)|amma|/->``->`/<1250,1000>[
      \J 0`
      \perp``
      \J 0;
      \j{0}``
      \jinv{0}`]

    \Dtriangle(0,500)|mmm|/`->`=/<1250,500>[
      \J\H \perp`
      \J\H\J 0`
      \J\H\J 0;`
      \J\H\jinv{0}`]

    \morphism(1250,1000)|m|<700,0>[
      \J\H\J 0`
      \J\H \perp;
      \J\H\j{0}]

    \morphism(1950,1000)|m|<550,-500>[
      \J\H \perp`
      \J 0;
      \J\h{\perp}]

    \place(1250,250)[(1)]
    \place(1500,750)[(2)]
    \place(380,1000)[(3)]
    \place(2000,1250)[(4)]
  \end{diagram}
  Diagram 1 commutes by naturality of $\varepsilon$, diagram 2
  commutes because $\varepsilon$ is the counit of the symmetric
  comonoidal adjunction, diagram 3 clearly commutes, and diagram 4
  commutes because $\j{0}$ is an isomorphism
  (Lemma~\ref{lemma:symmetric_comonoidal_isomorphisms}).
  
\item[Case.]\ \\
  \begin{diagram}
    \square|amma|<950,500>[
      \wn (\wn A \oplus \wn A)`
      \wn A \oplus \wn A`
      \wn\wn A`
      \wn A;
      u`
      \wn\c{}`
      \c{}`
      \mu]
  \end{diagram}
  This diagram commutes because the following fully expanded one does:
  \begin{center}
    \tiny
    \rotatebox{90}{$\bfig
      \square|amma|/->`->``/<1250,1500>[
        \J\H\J(\H A + \H A)`
        \J\H(\J\H A \oplus \J\H A)`
        \J\H\J\H A`;
        \J\H\j{}`
        \J\H\J\codiag{}``]
      \square(1250,0)|amma|/->``->`/<1250,1500>[
        \J\H(\J\H A \oplus \J\H A)`
        \J(\H\J\H A + \H\J\H A)``
        \J\H\J\H A;
        \J\h{}``
        \J\codiag{}`]
      \morphism/=/<2500,0>[\J\H\J\H A`\J\H\J\H A;]

      \Atriangle/<-`<-`=/<1250,500>[
        \J\H A`
        \J\H\J\H A`
        \J\H\J\H A;
        \J\varepsilon`
        \J\varepsilon`]

      \Vtriangle(0,1000)/`->`->/<1250,500>[
        \J\H\J(\H A + \H A)`
        \J(\H\J\H A + \H\J\H A)`
        \J(\H A + \H A);`
        \J\varepsilon`
        \J(\varepsilon + \varepsilon)]

      \morphism(1250,1000)|m|<0,-500>[
        \J(\H A + \H A)`
        \J\H A;
        \J\codiag{}]

      \square(0,1500)|amma|/->`->``/<1250,500>[
        \J\H(\J\H A \oplus \J\H A)`
        \J(\H\J\H A + \H\J\H A)`
        \J\H\J(\H A + \H A)`;
        \J\h{}`
        \J\H\jinv{}``]        

      \square(1250,1500)|amma|/->``->`/<1250,500>[
        \J(\H\J\H A + \H\J\H A)`
        \J\H\J\H A \oplus \J\H\J\H A`
        \J\H(\J\H A \oplus \J\H A)`
        \J(\H\J\H A + \H\J\H A);
        \j{}`
        `
        \jinv{}`]

      \square(2500,0)|amma|<1250,1500>[
        \J(\H\J\H A + \H\J\H A)`
        \J(\H A + \H A)`
        \J\H\J\H A`
        \J\H A;
        \J(\varepsilon + \varepsilon)`
        \J\codiag{}`
        \J\codiag{}`
        \J\varepsilon]

      \square(2500,1500)|amma|<1250,500>[
        \J\H\J\H A \oplus \J\H\J\H A`
        \J\H A \oplus \J\H A`
        \J(\H\J\H A + \H\J\H A)`
        \J(\H A + \H A);
        \J\varepsilon \oplus \J\varepsilon`
        \jinv{}`
        \jinv{}`
        \J(\varepsilon + \varepsilon)]

      \place(1250,250)[(1)]
      \place(625,750)[(2)]
      \place(1875,750)[(3)]
      \place(1250,1250)[(4)]
      \place(1250,1750)[(5)]
      \place(3125,1750)[(6)]
      \place(3125,750)[(7)]
      \efig$}
  \end{center}
\end{itemize}
Diagram 1 clearly commutes, diagram 2 commutes by naturality of
$\varepsilon$, diagram 3 commutes by naturality of $\codiag{}$,
diagram 4 commutes because $\varepsilon$ is the counit of the
symmetric comonoidal adjunction, diagram 5 commutes because $\j{}$
is an isomorphism
(Lemma~\ref{lemma:symmetric_comonoidal_isomorphisms}), diagram 6
commutes by naturality of $\jinv{}$, and diagram 7 is the same
diagram as 3, but this diagram is redundant for readability.
% subsection proof_of_lemma:right_weakening_and_contraction (end)

\subsection{Proof of Lemma~\ref{lemma:monoid-morphism}}
\label{sec:proof_of_lemma:monoid-morphism}
Suppose $\cat{L} : \func{H} \dashv \func{J} : \cat{C}$ is a dual LNL
model.  Then we know $\wn A = \J\H A$ is a symmetric comonoidal
monad by Lemma~\ref{lemma:symmetric_comonoidal_monad}.  Bellin
\cite{Bellin:2012} remarks that by Maietti, Maneggia de Paiva and
Ritter's Proposition~25 \cite{Maietti2005}, it suffices to show that
$\mu_A : \wn\wn A \mto \wn A$ is a monoid morphism.  Thus, the
following diagrams must commute:
\begin{itemize}
\item[Case.]\ \\
  \begin{diagram}
    \square|amma|<950,500>[
      \wn\wn A \oplus \wn\wn A`
      \wn\wn A`
      \wn A \oplus \wn A`
      \wn A;
      \c{\wn A}`
      \mu_A \oplus \mu_A`
      \mu_A`
      \c{A}]
  \end{diagram}
  This diagram commutes because the following fully expanded one
  does:
  \begin{diagram}
    \square|amma|<1000,500>[
      \J\H\J\H A \oplus \J\H\J\H A`
      \J(\H\J\H A + \H\J\H A)`
      \J\H A \oplus \J\H A`
      \J(\H A + \H A);
      \jinv{}`
      \J\varepsilon \oplus \J\varepsilon`
      \J(\varepsilon + \varepsilon)`
      \jinv{}]

    \square(1000,0)|amma|<1000,500>[
      \J(\H\J\H A + \H\J\H A)`
      \J\H\J\H A`
      \J(\H A + \H A)`
      \J\H A;
      \J\codiag{}`
      \J(\varepsilon + \varepsilon)`
      \J\varepsilon`
      \J\codiag{}]      
  \end{diagram}
  The left square commutes by naturality of $\jinv{}$ and the right
  square commutes by naturality of the codiagonal.
  
\item[Case.]\ \\
  \begin{diagram}
    \Atriangle|aaa|[
      \perp`
      \wn\wn A`
      \wn A;
      \w{\wn A}`
      \w{A}`
      \mu_A]
  \end{diagram}
  This diagram commutes because the following fully expanded one
  does:
  \begin{diagram}
    \square|amma|/=`->`->`=/<1000,500>[
      \perp`
      \perp`
      \J 0`
      \J 0;`
      \jinv{0}`
      \jinv{0}`]

    \square(0,-500)|amma|/=`->`->`->/<1000,500>[
      \J 0`
      \J 0`
      \J\H\J\H A`
      \J\H A;`
      \J\diamond`
      \J\diamond`
      \J\varepsilon]
  \end{diagram}
  The top square trivially commutes, and the bottom square commutes
  by uniqueness of the initial map.
\end{itemize}
% section proof_of_lemma:monoid-morphism (end)

\input{DLNL-proofs-output}

% section proofs (end)

%%% Local Variables: 
%%% mode: latex
%%% TeX-master: main.tex
%%% End: 
