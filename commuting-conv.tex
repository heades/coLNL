\begin{enumerate}
\item[] {\bf Non linear rules. }
\item \emph{disjunction intro} TC$_{+_{I_1} }$ and TC$_{+_{I_2} }$ commute upwards with every inference and the terms obtained are the same. 
\item \emph{disjunction  elim} TC$_{+_E }$ commutes upwrds with inferences in the derivation of the major premise, 
the terms assigned to the resulting subderivations are equated. For instance
\[
\AxiomC{$\hskip1.6in [[y:T2 |-TC P2, t1 : T4 + T5]]\hskip.4in$}
\noLine
\UnaryInfC{$[[x:S |-TC P1, t:T2 + T3]] \quad [[z:T3 |-TC P3, t2: T4 + T5]] \quad [[|P2| = |P3|]]$}
\UnaryInfC{$[[x:S |-TC P1, case t  of y.P2, z.P3]]\hskip.5in$}
\noLine
\UnaryInfC{$\hskip.8in [[case t of y.t1, z.t2]] : [[T4 + T5]]$}
\AxiomC{$[[v1:T4 |-TC P4]] \quad [[v2:T5 |-TC P5]], [[|P4| = |P5|]]$}
\BinaryInfC{$[[x:S |-TC P1, h(case t of y.P2, z.P3), case (case t of y.t1, z.t2) of v1.P4, v2.P5]]$}
\DisplayProof
\]
commmutes to 
\[
\AxiomC{$[[x:S |-TC P1, t:T2 + T3]]$} 
\AxiomC{$[[y:T2 |-TC P2, t1 : T4 + T5]] \quad [[v1:T4 |-TC P4]] \quad [[v2:T5 |-TC P5]] \quad [[|P4| = |P5|]]$}
\UnaryInfC{$[[y:T2 |-TC P2, case t1 of v1.P4, v2.P5]] \quad [[|P4| = |P5|]]$}
\noLine
\UnaryInfC{$[[z:T3 |-TC P3, t2: T4 + T5]] \quad [[v1:T4 |-TC P4]] \quad [[v2:T5 |-TC P5]]$}
\UnaryInfC{$[[z:T3 |-TC P3, case t2 of v1.P4, v2.P5]]$}
\BinaryInfC{$[[x:S |-TC P1,  case t of y.(P2,case t1 of v1.P4, v2.P5), z.(P3,case t2 of v1.P4, v2.P5)]]$}
\DisplayProof
\]
\begin{remark}If $[[t1]] = [[inl s1]]$ and $[[u2]] = [[inr s2]]$
 then after commutation 
\[
[[y:T2 |-TC P2, case inl s1 of v1.P4, v2.P5]] \rightsquigarrow_{\beta} [[y:T2 |-TC P2, [s1/v1]P4]] 
\]
\[
[[y:T3 |-TC P3, case inr s2 of v1.P4, v2.P5]] \rightsquigarrow_{\beta} [[y:T2 |-TC P3, [s1/v2]P5]]
\]
\end{remark}
% \end{enumerate}
\item Subtraction introduction TC$_{-_I}$ commutes upwards  with inferences in both branches with any inference $\mathcal{I}$:
\begin{center}
\begin{tabular}{ccc}
\AxiomC{$[[x:S |-TC t'1 : T1, P'1]]$}
\RightLabel{$\mathcal{I}$}
\UnaryInfC{$[[x:S |-TC t1: T1, P1]]$}
\AxiomC{$[[y:T2 |-TC  P2]]$}
\RightLabel{TC$_{-_I}$}
\BinaryInfC{$[[x:S |-TC P1, mkc(t,y) : T1 - T2,  [y(t)/y]P2]]$}
\DisplayProof 
&\quad 
&
\AxiomC{$[[x:S |-TC t1 : T1, P1]]$}
\AxiomC{$[[y:T2 |-TC P'2]]$}
\RightLabel{$\mathcal{I}$}
\UnaryInfC{$[[y:T2 |-TC P2]]$}
\RightLabel{TC$_{-_I}$}
\BinaryInfC{$[[x:S |-TC P1, mkc(t_1,y) : T1 - T2, [y(t)/y]P2]]$}
\DisplayProof \\
\\
commutes to & & commutes to \\
\\
\AxiomC{$[[x:S |-TC t1 : T1, P'1]]$}
\AxiomC{$[[y:T2 |-TC P2]]$}
\RightLabel{TC$_{-_I}$}
\BinaryInfC{$[[x:S |-TC P'1, mkc(t1,y) : T1 - T2, [y(t')/y]P2]]$}
\RightLabel{$\mathcal{I}$}
\UnaryInfC{$[[x:S |-TC P1, mkc(t1,y) : T1 - T2,  [y(t)/y]P2]]$}
\DisplayProof& &
\AxiomC{$[[x:S |-TC t1 : T1, P1]]$}
\AxiomC{$[[y:T2 |-TC P'2]]$}
\RightLabel{TC$_{-_I}$}
\BinaryInfC{$[[x:S |-TC P1, mkc(t1,y) : T1 - T2, P'2]]$}
\RightLabel{TC$_{-_I}$}
\RightLabel{$\mathcal{I}$}
\UnaryInfC{$[[x:S |-TC P1, mkc(t_1,y) : T1 - T2, P2]]$}
\DisplayProof
\end{tabular}
\end{center}
\item Subtraction elimination TC$_{-_E}$ commutes upwards. For instance, 
\begin{center}
\begin{tabular}{c}
\AxiomC{$[[x:S |-TC w:S1, P1]] \qquad [[z:S2 |-TC P2, t1:T1 - T2]]$}
\UnaryInfC{$[[x:S |-TC P1, mkc(w,z):S1 - S2, [z(w)/z]P2,[z(w)/z]t1: T1 - T2]]$}
\AxiomC{$[[y:T1 |-TC t:T2, P3]]$}
\BinaryInfC{$[[x:S |-TC P1, mkc(w,z):S1 - S2, [z(w)/z]P2, {postp(y -> t,[z(w)/z]t1)}, [y([z(w)/z]t1)/y]P3]]$}
\DisplayProof\\
\\
commutes to \\
\\
\AxiomC{$[[x:S |-TC w:S1, P1]]$}
\AxiomC{$[[z:S2 |-TC P2, t1:T1 - T2]] \qquad [[y:T1 |-TC t:T2, P3]]$}
\UnaryInfC{$[[z:S2 |-TC P2, {postp(y -> t, t1)}, [y(t1)/y]P3]]$}
\BinaryInfC{$[[x:S |-TC  P1, mkc(w,z):S1 - S2, [z(w)/z]P2, {postp(y -> t, [z(w)/z]t1)}, [z(w)/z][y(t1)/y]P3]]$}
\DisplayProof\\
\end{tabular}
\end{center}
where 
\begin{equation}
[[ [z(w)/z][y(t1)/y]P3]] =  [[ [y([z(w)/z]t1)/y]P3]]
\end{equation}
since $[[z]] \notin [[P2]]$.
\item[] {\bf Linear rules.}
\item The $\bot$ introduction rule TILL$_{\bot I}$ rule commutes with any inference, as $\mathtt{connect}_{\bot}
\mathtt{to} e$ can be ``rewired'' to any term in the context. 
\item The commutations of the rules for linear subtraction TILL$_{\lsub I}$ and TILL$_{\lsub E}$ are similar to those for non-linear subtraction. 
\item Linear disjunction (\emph{par}) introduction (TLL$_{\oplus I}$) commutes with any inference. 
Linear disjunction elimination  (TLL$_{\oplus E}$) also commutes upwards. For example (writing a proof without 
non-linear parts for simplicity) we have the following:
{\small
\begin{center}
\begin{tabular}{c}
\AxiomC{$[[x:A |-TL D, e:(B1 (+) B2)]]$}
\AxiomC{$[[y:B1 |-TL D1, e1:C1 (+) C2]] \qquad [[v:C1 |-TL D3]]\qquad [[w:C2 |-TL D4]]$}
\RightLabel{$\oplus$ E}
\UnaryInfC{$[[y:B1 |-TL D1, [casel e1/v]D3, [caser e1/w]D4]]$}
\noLine
\UnaryInfC{$[[z:B2 |-TL D2]] \hskip1.7in$}
\RightLabel{$\oplus$ E}
\BinaryInfC{$[[x:A |-TL D, [casel e/y]D1,  [casel e/y][casel e1/v]D3, [casel e/y][caser e1/w]D4, [caser e/z]D2]] $}
\DisplayProof\\
\\
commutes to\\
\\
\AxiomC{$[[x:A |-TL D, e:(B1 (+) B2)]] \qquad [[y:B1 |-TL D1, e1:C1 (+) C2]] \quad [[z:B2 |-TL D2]]$}
\RightLabel{$\oplus$ E}
\UnaryInfC{$[[x:A |-TL D, [casel e/y]D1, [casel e/y]e1:C1 (+) C2, [caser e/z]D2]]$}
\noLine
\UnaryInfC{$\hskip3in  [[v:C1 |-TL D3]] \qquad [[w:C2 |-TL D4]]$}
\RightLabel{$\oplus$ E}
\UnaryInfC{$[[x:A |-TL D,[casel e/y]D1,[casel [casel e/y]e1/v]D3,[caser [casel e/y]e1/w]D4,[caser e/z]D2]]$}
\DisplayProof
\end{tabular}
\end{center} }
% 
Now 
\begin{equation}
 [[ [casel e/y][casel e1/v]D3]] \quad = \quad  [[ [casel [casel e/y]e1/v]D3]]
\end{equation}
because $y$ does not occur in $[[D3]]$, only in $[[e1]]$ and 
\[
[[ [casel e/y][casel e1/w]D4]] \quad = \quad [[ [caser [casel e/y]e1/w]D4]]
\]
because $y$ does not occur in $[[D4]]$, only in $[[e1]]$.

\end{enumerate}
