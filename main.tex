\documentclass{lmcs}
\usepackage{amssymb,amsmath}
\usepackage{cmll}
\usepackage{txfonts}
\usepackage{graphicx}
\usepackage{stmaryrd}
\usepackage{todonotes}
\usepackage{mathpartir}
\usepackage{hyperref}
\usepackage{mdframed}
\usepackage[all]{xypic}
\usepackage[barr]{xy}

\newtheorem{theorem}{Theorem}
\newtheorem{lemma}[theorem]{Lemma}
\newtheorem{corollary}[theorem]{Corollary}
\newtheorem{definition}[theorem]{Definition}
\newtheorem{proposition}[theorem]{Proposition}
\newtheorem{example}[theorem]{Example}
\newtheorem{remark}[theorem]{Remark}

%% This renames Barr's \to to \mto.  This allows us to use \to for imp
%% and \mto for a inline morphism.
\let\mto\to
\let\to\relax
\newcommand{\to}{\rightarrow}
\newcommand{\ndto}[1]{\to_{#1}}
\newcommand{\ndwedge}[1]{\wedge_{#1}}

% Commands that are useful for writing about type theory and programming language design.
%% \newcommand{\case}[4]{\text{case}\ #1\ \text{of}\ #2\text{.}#3\text{,}#2\text{.}#4}
\newcommand{\interp}[1]{\llbracket #1 \rrbracket}
\newcommand{\normto}[0]{\rightsquigarrow^{!}}
\newcommand{\join}[0]{\downarrow}
\newcommand{\redto}[0]{\rightsquigarrow}
\newcommand{\nat}[0]{\mathbb{N}}
\newcommand{\fun}[2]{\lambda #1.#2}
\newcommand{\CRI}[0]{\text{CR-Norm}}
\newcommand{\CRII}[0]{\text{CR-Pres}}
\newcommand{\CRIII}[0]{\text{CR-Prog}}
\newcommand{\subexp}[0]{\sqsubseteq}
%% Must include \usepackage{mathrsfs} for this to work.

\date{}

\let\b\relax
\let\d\relax
\let\t\relax
\let\r\relax
\let\c\relax
\let\j\relax
\let\wn\relax
\let\H\relax

% Cat commands.
\newcommand{\powerset}[1]{\mathcal{P}(#1)}
\newcommand{\cat}[1]{\mathcal{#1}}
\newcommand{\func}[1]{\mathsf{#1}}
\newcommand{\iso}[0]{\mathsf{iso}}
\newcommand{\H}[0]{\func{H}}
\newcommand{\J}[0]{\func{J}}
\newcommand{\catop}[1]{\cat{#1}^{\mathsf{op}}}
\newcommand{\Hom}[3]{\mathsf{Hom}_{\cat{#1}}(#2,#3)}
\newcommand{\limp}[0]{\multimap}
\newcommand{\colimp}[0]{\multimapdotinv}
\newcommand{\dial}[1]{\mathsf{Dial_{#1}}(\mathsf{Sets^{op}})}
\newcommand{\dialSets}[1]{\mathsf{Dial_{#1}}(\mathsf{Sets})}
\newcommand{\dcSets}[1]{\mathsf{DC_{#1}}(\mathsf{Sets})}
\newcommand{\sets}[0]{\mathsf{Sets}}
\newcommand{\obj}[1]{\mathsf{Obj}(#1)}
\newcommand{\mor}[1]{\mathsf{Mor(#1)}}
\newcommand{\id}[0]{\mathsf{id}}
\newcommand{\lett}[0]{\mathsf{let}\,}
\newcommand{\inn}[0]{\,\mathsf{in}\,}
\newcommand{\cur}[1]{\mathsf{cur}(#1)}
\newcommand{\curi}[1]{\mathsf{cur}^{-1}(#1)}
\newcommand{\m}[1]{\mathsf{m}_{#1}}
\newcommand{\n}[1]{\mathsf{n}_{#1}}
\newcommand{\b}[1]{\mathsf{b}_{#1}}
\newcommand{\d}[1]{\mathsf{d}_{#1}}
\newcommand{\h}[1]{\mathsf{h}_{#1}}
\newcommand{\p}[1]{\mathsf{p}_{#1}}
\newcommand{\q}[1]{\mathsf{q}_{#1}}
\newcommand{\t}[0]{\mathsf{t}}
\newcommand{\r}[1]{\mathsf{r}_{#1}}
\newcommand{\s}[1]{\mathsf{s}_{#1}}
\newcommand{\w}[1]{\mathsf{w}_{#1}}
\newcommand{\c}[1]{\mathsf{c}_{#1}}
\newcommand{\j}[1]{\mathsf{j}_{#1}}
\newcommand{\jinv}[1]{\mathsf{j}^{-1}_{#1}}
\newcommand{\wn}[0]{\mathop{?}}
\newcommand{\codiag}[1]{\bigtriangledown_{#1}}

\newenvironment{changemargin}[2]{%
  \begin{list}{}{%
    \setlength{\topsep}{0pt}%
    \setlength{\leftmargin}{#1}%
    \setlength{\rightmargin}{#2}%
    \setlength{\listparindent}{\parindent}%
    \setlength{\itemindent}{\parindent}%
    \setlength{\parsep}{\parskip}%
  }%
  \item[]}{\end{list}}

\let\diagram\relax
\newenvironment{diagram}{
  \begin{center}
    \begin{math}
      \bfig
}{
      \efig
    \end{math}
  \end{center}
}

\input{bussproofs.sty}
\def\limp{\mathrel{-\!\circ}}
\def\lsub{\mathrel{\bullet\!-}}

%% Ott
\input{DualLNLLogic-inc}
\renewcommand{\DualLNLLogicdrule}[4][]{{\displaystyle\frac{\begin{array}{l}#2\end{array}}{#3}\,\DualLNLLogicdrulename{#4}}}
\renewcommand{\DualLNLLogicdrulename}[1]{#1}

\renewcommand{\DualLNLLogicdruleCXXidName}{\text{C\_}\text{id}}
\renewcommand{\DualLNLLogicdruleCXXwkName}{\text{C\_}\text{weak}}
\renewcommand{\DualLNLLogicdruleCXXcrName}{\text{C\_}\text{contr}}
\renewcommand{\DualLNLLogicdruleCXXexName}{\text{C\_}\text{ex}}
\renewcommand{\DualLNLLogicdruleCXXfLName}{\text{C\_}0}
\renewcommand{\DualLNLLogicdruleCXXdLName}{\text{C\_}+_L}
\renewcommand{\DualLNLLogicdruleCXXdROneName}{\text{C\_}+_{R_1}}
\renewcommand{\DualLNLLogicdruleCXXdRTwoName}{\text{C\_}+_{R_2}}
\renewcommand{\DualLNLLogicdruleCXXsLName}{\text{C\_}-_L}
\renewcommand{\DualLNLLogicdruleCXXsRName}{\text{C\_}-_R}
\renewcommand{\DualLNLLogicdruleCXXcutName}{\text{C\_}\text{cut}}
\renewcommand{\DualLNLLogicdruleCXXhLName}{\mathsf{H}_L}
\renewcommand{\DualLNLLogicdruleCXXmcutName}{\text{C\_cut}_n}
\renewcommand{\DualLNLLogicdruleCXXadLName}{}

\renewcommand{\DualLNLLogicdruleLXXidName}{\text{LL\_id}}
\renewcommand{\DualLNLLogicdruleLXXwkName}{\text{LC\_weak}}
\renewcommand{\DualLNLLogicdruleLXXctrName}{\text{LC\_contr}}
\renewcommand{\DualLNLLogicdruleLXXexName}{\text{LL\_ex}}
\renewcommand{\DualLNLLogicdruleLXXCexName}{\text{LC\_ex}}
\renewcommand{\DualLNLLogicdruleLXXcutName}{\text{LL\_cut}}
\renewcommand{\DualLNLLogicdruleLXXCcutName}{\text{LC\_cut}}
\renewcommand{\DualLNLLogicdruleLXXflLName}{\text{LL\_}\hspace{-3px}\perp_L}
\renewcommand{\DualLNLLogicdruleLXXflRName}{\text{LL\_}\hspace{-3px}\perp_R}
\renewcommand{\DualLNLLogicdruleLXXdROneName}{\text{LC\_}+_{R_1}}
\renewcommand{\DualLNLLogicdruleLXXdRTwoName}{\text{LC\_}+_{R_2}}
\renewcommand{\DualLNLLogicdruleLXXpLName}{\text{LL\_}\oplus_L}
\renewcommand{\DualLNLLogicdruleLXXpRName}{\text{LL\_}\oplus_R}
\renewcommand{\DualLNLLogicdruleLXXsLName}{\text{LL\_}\hspace{-3px}\colimp_L}
\renewcommand{\DualLNLLogicdruleLXXsRName}{\text{LL\_}\hspace{-3px}\colimp_R}
\renewcommand{\DualLNLLogicdruleLXXCsRName}{\text{LL\_}-_R}
\renewcommand{\DualLNLLogicdruleLXXjLName}{\func{J}_L}
\renewcommand{\DualLNLLogicdruleLXXjRName}{\func{J}_R}
\renewcommand{\DualLNLLogicdruleLXXhRName}{\func{H}_R}
\renewcommand{\DualLNLLogicdruleLXXCmcutName}{\text{LC\_cut}_n}

\renewcommand{\DualLNLLogicdruleNCXXidName}{\text{NC\_}\text{id}}
\renewcommand{\DualLNLLogicdruleNCXXzEName}{\text{NC\_}0_E}
\renewcommand{\DualLNLLogicdruleNCXXdIOneName}{\text{NC\_}+_{I_1}}
\renewcommand{\DualLNLLogicdruleNCXXdITwoName}{\text{NC\_}+_{I_2}}
\renewcommand{\DualLNLLogicdruleNCXXdEName}{\text{NC\_}+_E}
\renewcommand{\DualLNLLogicdruleNCXXsubIName}{\text{NC\_}-_I}
\renewcommand{\DualLNLLogicdruleNCXXsubEName}{\text{NC\_}-_E}
\renewcommand{\DualLNLLogicdruleNCXXHEName}{\text{NC\_}\func{H}_E}
\renewcommand{\DualLNLLogicdruleNCXXweakName}{\text{NC\_}\text{weak}}
\renewcommand{\DualLNLLogicdruleNCXXcontrName}{\text{NC\_}\text{contr}}
\renewcommand{\DualLNLLogicdruleNCXXcutName}{\text{NC\_}\text{cut}}

\renewcommand{\DualLNLLogicdruleNLXXidName}{\text{NLL\_}\text{id}}
\renewcommand{\DualLNLLogicdruleNLXXpIName}{\text{NLL\_}\hspace{-3px}\perp_I}
\renewcommand{\DualLNLLogicdruleNLXXpEName}{\text{NLL\_}\hspace{-3px}\perp_E}
\renewcommand{\DualLNLLogicdruleNLXXparIName}{\text{NLL\_}\oplus_I}
\renewcommand{\DualLNLLogicdruleNLXXparEName}{\text{NLL\_}\oplus_E}
\renewcommand{\DualLNLLogicdruleNLXXsubIName}{\text{NLL\_}\hspace{-3px}\colimp_I}
\renewcommand{\DualLNLLogicdruleNLXXsubEName}{\text{NLL\_}\hspace{-3px}\colimp_E}
\renewcommand{\DualLNLLogicdruleNLXXJIName}{\text{NLL\_}\func{J}_I}
\renewcommand{\DualLNLLogicdruleNLXXJEName}{\text{NLL\_}\func{J}_E}
\renewcommand{\DualLNLLogicdruleNLXXHIName}{\text{NLL\_}\func{H}_I}
\renewcommand{\DualLNLLogicdruleNLXXHEName}{\text{NLL\_}\func{H}_E}
\renewcommand{\DualLNLLogicdruleNLXXweakName}{\text{NLC\_}\text{weak}}
\renewcommand{\DualLNLLogicdruleNLXXcontrName}{\text{NLC\_}\text{contr}}
\renewcommand{\DualLNLLogicdruleNLXXCcutName}{\text{NLC\_}\text{cut}}
\renewcommand{\DualLNLLogicdruleNLXXcutName}{\text{NLL\_}\text{cut}}

\renewcommand{\DualLNLLogicdruleTCXXidName}{\text{TC\_}\text{id}}
\renewcommand{\DualLNLLogicdruleTCXXweakName}{\text{TC\_}\text{weak}}
\renewcommand{\DualLNLLogicdruleTLXXweakName}{\text{TL\_}\text{weak}}
\renewcommand{\DualLNLLogicdruleTCXXzIName}{\text{TC\_}0_E}
\renewcommand{\DualLNLLogicdruleTCXXdIOneName}{\text{TC\_}+_{I_1}}
\renewcommand{\DualLNLLogicdruleTCXXdITwoName}{\text{TC\_}+_{I_2}}
\renewcommand{\DualLNLLogicdruleTCXXdEName}{\text{TC\_}+_E}
\renewcommand{\DualLNLLogicdruleTCXXsubIName}{\text{TC\_}-_I}
\renewcommand{\DualLNLLogicdruleTCXXsubEName}{\text{TC\_}-_E}
\renewcommand{\DualLNLLogicdruleTCXXHEName}{\text{TC\_}\func{H}_E}
%% \renewcommand{\DualLNLLogicdruleTCXXweakName}{\text{TC\_}\text{weak}}
\renewcommand{\DualLNLLogicdruleTCXXcontrName}{\text{TC\_}\text{contr}}
\renewcommand{\DualLNLLogicdruleTCXXcutName}{\text{TC\_}\text{cut}}

\renewcommand{\DualLNLLogicdruleTLXXidName}{\text{TLL\_}\text{id}}
\renewcommand{\DualLNLLogicdruleTLXXpIName}{\text{TLL\_}\hspace{-3px}\perp_I}
\renewcommand{\DualLNLLogicdruleTLXXpEName}{\text{TLL\_}\hspace{-3px}\perp_E}
\renewcommand{\DualLNLLogicdruleTLXXparIName}{\text{TLL\_}\oplus_I}
\renewcommand{\DualLNLLogicdruleTLXXparEName}{\text{TLL\_}\oplus_E}
\renewcommand{\DualLNLLogicdruleTLXXsubIName}{\text{TLL\_}\hspace{-3px}\colimp_I}
\renewcommand{\DualLNLLogicdruleTLXXsubEName}{\text{TLL\_}\hspace{-3px}\colimp_E}
\renewcommand{\DualLNLLogicdruleTLXXJIName}{\text{TLL\_}\func{J}_I}
\renewcommand{\DualLNLLogicdruleTLXXJEName}{\text{TLL\_}\func{J}_E}
\renewcommand{\DualLNLLogicdruleTLXXHIName}{\text{TLL\_}\func{H}_I}
\renewcommand{\DualLNLLogicdruleTLXXHEName}{\text{TLL\_}\func{H}_E}
%% \renewcommand{\DualLNLLogicdruleTLXXweakName}{\text{TLC\_}\text{weak}}
\renewcommand{\DualLNLLogicdruleTLXXcontrName}{\text{TLC\_}\text{contr}}
\renewcommand{\DualLNLLogicdruleTLXXCcutName}{\text{NLC\_}\text{cut}}
\renewcommand{\DualLNLLogicdruleTLXXcutName}{\text{TLL\_}\text{cut}}

\newcommand{\DLNLP}{\text{DLNL}^+}

\urldef{\mailsa}\path|{heades}@augusta.edu|

\begin{document}

\title{A Cointuitionistic Adjoint Logic}
\author{Harley Eades III}
\email{heades@augusta.edu}
\address{Computer Science, Augusta University, Augusta, GA}

\author{Gianluigi Bellin}
\email{gianluigi.bellin@univr.it}
\address{Dipartimento di Informatica, Universit\`{a} di Verona, Strada Le Grazie, 37134 Verona, Italy}

\maketitle 

\begin{abstract}

  

\end{abstract}

\section{Introduction}
\label{sec:introduction}
Bi-intuitionistic logic (BINT) is a conservative extension of
intuitionistic logic with perfect duality.  That is, BINT contains the
usual intuitionistic logical connectives such as true, conjunction,
and implication, but also their duals false, disjunction, and
coimplication. One leading question with respect to BINT is, what does
BINT look like across the three arcs -- logic, typed
$\lambda$-calculi, and category theory -- of the Curry-Howard-Lambek
correspondence?  A non-trivial (does not degenerate to a poset)
categorical model of BINT is currently an open problem.  This paper is
the first of two that will provide an answer to this open problem.

BINT can be seen as a mixing of two worlds: the first being
intuitionistic logic (IL), which is modeled categorically by a
cartesian closed category (CCC), and the second being the dual to
intuitionistic logic called cointuitionistic logic (coIL), which is
modeled by a cocartesian coclosed category (coCCC).  Crolard
\cite{Crolard:2001} showed that combining these two categories into
the same category results in it degenerating to a poset, that is,
there is at most one morphism between any two objects; we review this
result in
Section~\ref{subsec:cartesian_closed_and_cocartesian_coclosed_categories}.
However, this degeneration does not occur when both logics are linear.
We propose that these two worlds need to be separated, and then mixed
in a control way using the modalities from linear logic.  This
separation can be ultimately achieved by an adjoint formalization of
bi-intuitionistic logic.  This formalization consists of three worlds
instead of two: the first is intuitionistic logic, the second is
linear bi-intuitionistic (Bi-ILL), and the third is cointuitionistic
logic.  They are then related via two adjunctions as depicted by the
following diagram:
\begin{center}
  \begin{tikzpicture}
    \node (img) {\includegraphics[scale=0.4]{introDiag}};
    \node (dv) at (-2.3, 0.0) {\huge $\dashv$};
    \node (IL) at (-3.58, -0.1) {IL};
    \node (vd) at (2.3, 0.0) {\huge $\vdash$};
    \node (coIL) at (3.65, -0.1) {coIL};

    \node (ILL) at (-0.67, 0.0) {ILL};
    \node (coILL) at (0.71, 0.0) {coILL};
    \node (BiILL) at (0, 1.0) {Bi-ILL};
  \end{tikzpicture}
    
\end{center}
The adjoint between IL and ILL is known as a LNL model of ILL, and is
due to Benton \cite{Benton:1994}.  However, the dual to LNL models
which would amount to the adjoint between coILL and coIL has yet to
appear in the literature.

The main contribution of this paper is the definition and study of the
dual to Benton's LNL models as models of cointuitionistic logic called
dual LNL models.  Bellin \cite{Bellin:2012} was the first to propose
the dual to Bierman's \cite{Bierman:1994} linear categories which he
names dual linear categories as a model of cointuitionistic linear
logic.  We conduct a similar analysis to that of Benton for dual LNL
models by showing that dual LNL models are dual linear categories
(Section~\ref{subsec:dual_lnl_model_implies_dual_category}), and that
from a dual linear category we may obtain a dual LNL model
(Section~\ref{subsec:dual_category_implies_dual_lnl_model}).
Following this we give the definition of bi-LNL models by combining
our dual LNL models with Benton's LNL models to obtain a categorical
model of bi-intuitionistic logic
(Section~\ref{subsec:a_mixed_bi-linear_non-linear_model}), but we
leave its analysis and corresponding logic to a future paper.
Finally, we give the definition of dual LNL logic with a term
assignment (Section~\ref{sec:dual_lnl_logic} and
Section~\ref{sec:dual_lnl_term_assignment} respectively).

% section introduction (end)

\section{The Adjoint Model}
\label{sec:adjoint_model}
\input{categorical-model}
% section the_categorical_model (end)

\section{Dual LNL Logic}
\label{sec:dual_lnl_logic}
\input{dualLNL-logic-output}
% section dual_lnl_logic (end)


\section{Related and Future Work}
\label{sec:related_work}
The most comprehensive treatment of ILL is in Gavin Bierman's thesis
\cite{Bierman:1994}.  There one finds the Proof Theory (Chapter 2),
i..e, the sequent calculus with cut-elimination, natural deduction and
axiomatic versions of ILL. Then (Chapter 3) a term assignment to the
natural deduction and to the sequent calculus versions are presented
with $\beta$-reductions and commutative conversions, and strong
normalization and confluence are proved for the resulting calculus. A
painstaking analysis of the rules of the labeled calculus leads to
the construction of a categorical model of ILL, in particular of the
exponential part, a main contribution of Bierman and of the Cambridge
school of the 1990s with respect to previous models by Seely and
Lafont.  Bellin \cite{Bellin:2014} presents a categorical model of
co-intuitionistic linear logic based on a dualization of Bierman
\cite{Bierman:1994} construction for ILL..

% <<<<<<< HEAD
The most comprehensive treatment of ILL is in Gavin Bierman's thesis  \cite{Bierman:1994}.
There one finds the Proof Theory (Chapter 2), i.e, the sequent calculus with cut-eliminaton, natural deduction and
axiomatic versions of ILL. Then (Chapter 3) a term assignment to the natural deduction and to the sequent calculus
versions are presented with $\beta$-reductions and commutative conversions, and strong normalization and confluence
are proved for the resulting calculus. A painstaking analysis of the rules of the labeled calculus leads to the construction
of a categorical model of ILL, in particular of the exponential part, a main contribution of Bierman and of the
Cambridge school of the 1990s with respect to previous models by Seely and Lafont.
Bellin \cite{Bellin:2012} presents a categorical model of co-intuitionistic linear logic based on a dualization of
Bierman \cite{Bierman:1994} construction for ILL..

Benton's work \cite{Benton:1994} on LNL logic, which we read in a TeX report, presents the categorical model 
for Linear-Non-Linear Intuitionistic logic LNL. Chapter 2 shows
how to obtain a LNL model from a Linear Category and viceversa. Versions of the sequent calculus for LNL
are considered  and cut-elimination is proved for one calculus. Then Natural Deduction is given with term
assignment and the categorical interpretation of a fragment of the natural deduction system. $\beta$ reductions
and commuting conversions are presented.
The present work follows Benton's paper aiming at a (non-trivial) dualization of it.

Bi-intuitionistic logic was introduced by C.Rauszer \cite{Rauszer:1974},  \cite{Rauszer:1974a}, \cite{Rauszer:1977} with
an algebraic and Kripke semantics and  \cite{Rauszer:1974a} a Gentzen style sequent calculus. Co-intuitionistic logic requires a multiple conclusion system, because of the cotensor in the linear case and of contraction right in the non-linear one. This raises the problem of the relations between intuitionistic implication and disjunction, and, dually, 
between subtraction and conjunction. Maheara and Deagalin's restriction  
 that only one formula occures in the succedent [antecedent] of the premise of an implication right [subtraction left] yields a calculus that does not satisfies cut-elimination already in the logic FILL that extends ILL with the cotensor \emph{par} 
as noticed by Schellinx \cite{Schellinx:1991}.

Benton's work \cite{Benton:1994} on LNL logic, which we read in a TeX
report, presents the categorical model for Linear-Non-Linear
Intuitionistic logic LNL. Chapter 2 shows how to obtain a LNL model
from a Linear Category and vice versa. Versions of the sequent calculus
for LNL are considered and cut-elimination is proved for one
calculus. Then Natural Deduction is given with term assignment and the
categorical interpretation of a fragment of the natural deduction
system. $\beta$ reductions and commuting conversions are presented.
The present work follows Benton's paper aiming at a (non-trivial)
dualization of it.

Bi-intuitionistic logic was introduced by C.Rauszer
\cite{Rauszer:1974}, \cite{Rauszer:1974a}, \cite{Rauszer:1977} with an
algebraic and Kripke semantics and \cite{Rauszer:1974a} a Gentzen
style sequent calculus. Co-intuitionistic logic requires a multiple
conclusion system, because of the tìco-tensor in the linear case and
of contraction right in the non-linear one. This raises the problem of
the relations between intuitionistic implication and disjunction, and,
dually, between subtraction and conjunction. Maheara and Deagalin's
restriction that only one formula occurs in the succedent
[antecedent] of the premise of an implication right [subtraction left]
yields a calculus that does not satisfies cut-elimination already in
the logic FILL that extends ILL with the cotensor \emph{par} as
noticed by Schellinx \cite{Schellinx:1991}.
>>>>>>> 9fbc5b8c7473d9b3d8c55af93a8f19afad83c78d
\begin{center}
\begin{tabular}{ccc}
\AxiomC{$\Gamma, A \vdash B$}
\RightLabel{$\limp$ R}
\UnaryInfC{$\Gamma \vdash A \limp B$}
\DisplayProof & \hskip1in\strut& 
\AxiomC{$ A \vdash B, \Delta$}
\RightLabel{$\lsub$ E}
\UnaryInfC{$A \lsub B \vdash \Delta$}
\DisplayProof 
\end{tabular}
\end{center}
As a counterexample, consider the sequent $p \Rightarrow q, r
\rightarrow ((p - q) \wedge r)$ given by Pinto and Uusatlu around 2003
\cite{Pinto-Uustalu:2010}, which is provable with cut but not cut-free
with Dragalin's restrictions.

Hyland and De Paiva introduced a sequent calculus for FILL labeled
with terms
\begin{center}
\begin{tabular}{c}
\AxiomC{$\overline{y}:\Gamma, x:A \vdash t:B, \overline{u}:\Delta$}
\RightLabel{$\limp$ R}
\UnaryInfC{$\overline{y}: \Gamma \vdash \lambda x:T A \limp B, \overline{u}:\Delta$}
\DisplayProof
\end{tabular} 
\end{center}
% <<<<<<< HEAD
where $x: A$ occurs in $t:B$ if and only if there is an ``essential dependency'' of $B$ from $A$. 
The restriction on the $\limp$ I is that $x$ does not occur in the terms  $\overline{u}:\Delta$.
The original term assignment did not guarantee cut-elimination, as noticed by Bierman \cite{Bierman:1996}; 
the assignment to \emph{par left} ($\oplus$ L)  had to be fine tuned, as indicated by Bellin \cite{Bellin:1997} 
\footnote{For \emph{par} left ($\oplus L$) we need the following, :
where $x: A$ occurs in $t:B$ if and only if there is an ``essential
dependency'' of $B$ from $A$.  The restriction on the $\limp$ I is
that $x$ does not occur in the terms $\overline{u}:\Delta$.  The
original term assignment did no guarantee cut-elimination, as noticed
by Bierman \cite{Bierman:1996}; the assignment to \emph{par left}
($\oplus$ L) had to be fine tuned, as indicated by Bellin
\cite{Bellin:1997}
\footnote{For \emph{par} left ($\oplus L$) we need the following:
% >>>>>>> 9fbc5b8c7473d9b3d8c55af93a8f19afad83c78d
\[
\AxiomC{$x:A, \overline{x}:\Gamma\vdash \overline{r}: \Delta\qquad y:B, \overline{y}:\Pi\vdash \overline{s}: \Lambda$}
\UnaryInfC{$z: A \oplus B, \overline{x}:\Gamma, \overline{y}:\Pi\vdash  \overline{r'}: \Delta, 
\overline{s'}: \Lambda $}
\DisplayProof
\]
where for $r'_i\in \overline{r'}$ and $s'_j \in \overline{s'}$ we have 
\[
\begin{tabular}{rll}
$r'_i =$ & $\mathtt{let}\ z\ \mathtt{be}\ x-\ \mathtt{in}\ r_i, \qquad$ & if $x$ occurs in $r_i$, \\
             & $r_i$                                                                              & otherwise.\\
$s'_j =$ & $\mathtt{let}\ z\ \mathtt{be}\ -y\ \mathtt{in}\ s_j, \qquad$ & if $y$ occurs in $s_j$, \\
             & $s_j$                                                                              & otherwise.\\
\end{tabular}
\]
We may introduce non-existent dependencies if we define always $r'_i =
\mathtt{let}\ z\ \mathtt{be}\ x-\ \mathtt{in}\ r_i.$
}

A detailed presentation of the term calculus for FILL with a full proof of cut elimination, is in \cite{EadesDePaiva2016}, 
where the correctness for a categorical semantics for FILL is also proved.
Another correct formalization of FILL, a sequent calculus with a relational annotation, was given by Bra\"uner and De Paiva \cite{BraunedDePaiva:1997}, with a proof of cut-elimination.
The second author \cite{Bellin:1997} gave a system of proof nets for FILL which sequentialize in the sequent calculus
with term assignment; the essential fact here is that $x:A$ occurs in $t:B$ if and only if there is a ``directed chain''
between $A$ and $B$ in the proof structure. Here cut elimination is proved by reduction to cut-elimination for proof nets.  

A detailed presentation of the term calculus for FILL with a full
proof of cut elimination, is in \cite{EadesDePaiva2016}, where the
correctness for a categorical semantics for FILL is also proved.
Another correct formalization of FILL, a sequent calculus with a
relational annotation, was given by Bra\"uner and De Paiva
\cite{BraunedDePaiva:1997}, with a proof of cut-elimination.  The
second author \cite{Bellin1997} gave a system of proof nets for FILL
which sequentialize in the sequent calculus with term assignment; the
essential fact here is that $x:A$ occurs in $t:B$ if and only if there
is a ``directed chain'' between $A$ and $B$ in the proof
structure. Here cut elimination is proved by reduction to
cut-elimination for proof nets.


A system of two-sided proof nets (in the style of natural deduction)
was given by Cockett and Seely \cite{Cockett:1997}.  For
Bi-Intuitionistic Linear Logic, they gave also a system of proof nets,
corresponding to a sequent calculus without annotations and
restrictions that therefore collapses into classical MLL.  Recently,
Clouston, Dawson, Goré and Tiu cite{Clouston} gave an annotation-free
formalization for BILL, alternative to sequent calculi, in the form of
deep-inference and display calculi for BILL. This calculus enjoys
cut-elimination and is relevant to the categorical semantics
bi-intuitionistic linear logic.
 

Tristan Crolard \cite{Crolard:2001,Crolard:2004} made an in-depth study of Rauszer's logic. In \cite{Crolard:2001} he showed that models of Rauszer logic (called ``subtractive logic'') based on bi-cartesian closed categories (with co-exponents) collapse to preorders.
He also studied models of subtractive logic and showed that its first order theory is constant-domain logic, thus it is not 
a conservative extension of intuitionistic logic.

Crolard \cite{Crolard:2004} develops the type theory for subtractive logic, extending a system of multiple conclusion
classical natural deduction with a connective of subtraction and then decorating proofs with a system of \cite{annotations
of dependencies} that allow to identify ``constructive proofs'': these are derivations where only the premise of an implication introduction depends on the discharged assumption and only the premise of a subtraction elimination depends on the discharged conclusion. Therefore Crolard's sequent calculus with annotations is not affected by the 
counterexamples to cut-eliminations. 

The type theory is Parigot $\lambda\mu$-calculus extended with operators for sums, products and subtraction. 
where the operators for subtraction introduction and elimination are understood as a calculus of co-routines.
A constructive system of co-routines is then obtained by imposing restrictions on terms corresponding to the restrictions
 on constructive proofs.  Annotation-free formalizations use the display calculus \cite{Gore:2000}, nested sequents
\cite{GorePostnieceTiu:2008} and deep inference \cite{Postniece:2009}.

Tristan Crolard \cite{Crolard:2001,Crolard:2004} made an in-depth
study of Rauszer's logic. In \cite{Crolard:2001} he showed that models
of Rauszer logic (called ``subtractive logic'') based on bi-cartesian
closed categories (with co-exponents) collapse to prerders.  He than
studied models of subtractive logic and showed that its first order
theory is constant-domain logic, thus it is not a conservative
extension of intuitionistic logic.  Crolard \cite{Crolard:2004}
develops the type theory for subtractive logic, extending a system of
multiple conclusion classical natural deduction with a connective of
subtraction and then decorating proofs with a system of
\cite{annotationsofdependencies} that allow to identify ``constructive
proofs'': these are derivations where only the premise of an
implication introduction depends on the discharged assumption and only
the premise of a subtraction elimination depends on the discharged
conclusion. Therefore Crolard's sequent calculus with annotations is
not affected by the counterexamples to cut-eliminations.

The type theory is Parigot $\lambda\mu$-calculus extended with
operators for sums, products and subtraction.  where the operators for
subtraction introduction and elimination are understood as a calculus
of co-routines.  A constructive system of co-routines is then obtained
by imposing restrictions on terms corresponding to the restrictions on
constructive proofs.  Annotation-free formalizations use the display
calculus \cite{Gore:2000}, nested sequents \cite{GorePostnieceTiu:2008}
and deep inference \cite{Postniece:2009}.

In a series of papers the second author gave a ``pragmatic''
interpretation interpretation of bi-intuitionism, where intuitionistic
and co-intuitionistic logic are interpreted as logics of the acts of
assertion and making a hypothesis, respectively, the interactions
between the two sides depending on negations (see, e.g,
\cite{Bellin:2014} Assertions, hypotheses, conjectures, expectations:
Rough-sets semantics and proof-theory.)  Here a term assignment for
co-intuitionistic logic has been derived from Crolard's definition but
independently from the $\lambda\mu$ framework. This calculus was used
here as a term assignment of Dual LNL logic.


To achieve the project outlined in the introduction of putting
together intuitionistic and co-intuitionistic adjoint logic in the
environment of BILL the definition of a suitable syntax for BILL will
play a role.


% More generally, notice also that co-intuitionistic logic also arises as a logic of refutations of intuitionistic sequents,
% dual to an intuitionistic logic of proofs \cite{Tranchini}.
% However, co-intuitionistic logic could also be regarded as \emph{opposite} to intuitionistic logic, rather than its dual.
% =======
% More generally, notice also that co-intuitionistic logic also arises
% as a logic of refutations of intuitionistic sequents, dual to an
% intuitionistic logic of proofs \cite{Tranchini}.  However,
% co-intuitionistic logic could also be regarded as \emph{opposite} to
% intuitionistic logic, rather than its dual.

%%% Local Variables: 
%%% mode: latex
%%% TeX-master: main.tex
%%% End: 
% >>>>>>> 9fbc5b8c7473d9b3d8c55af93a8f19afad83c78d


% The most comprehensive treatment of ILL is in Gavin Bierman's thesis  \cite{Bierman:1994}. 
% There one finds the Proof Theory (Chapter 2), i..e, the sequent calculus with cut-eliminaton, natural deduction and 
% axiomatic versions of ILL. Then (Chapter 3) a term assignment to the natural deduction and to the sequent calculus
% versions are presented with $\beta$-reductions and commutative conversions, and strong normalization and 
% confluence are proved for the resulting calculus. A painstaking analysis of the rules of the labelled calculus 
% leads to the construction of a categorical model of ILL, in particular of the exponential part, a main contribution 
% of Bierman and of the Cambridge school of the 1990s with respect to previous models by Seely and Lafont. 
% Bellin \cite{Bellin:2014} presents a categorical model of co-intuitionistic linear logic based on a dualization of 
% Bierman \cite{Bierman:1994} construction for ILL..

% Benton's work \cite{Benton:1994} on LNL logic, which we read in a TeX report less systematc and detailed 
% than Bierman's, presents the categorical model Linear-Non-Linear Intuitionistic logic LNL. Chapter 2 shows 
% how to obtain a LNL model from a Linear Category and viceversa. Versions of the sequent calculus for LNL 
% are considered  and cut-elimination is proved for one calculus. Then Natural Deduction is given with term 
% assignment and the categorical interpretaiton of a fragment of the natural deduction system. $\beta$ reductions 
% and commuting conversions are presented.
% The present work follows Benton's paper aiming at a (non-trivial) dualization of it.   
% section related_work (end)

\section{Conclusion}
\label{sec:conclusion}
TODO
% section conclusion (end)

\bibliographystyle{plainurl} \bibliography{ref}

\appendix
\section{Proofs}
\label{sec:proofs}

\subsection{Proof of Lemma~\ref{lemma:symmetric_comonoidal_isomorphisms}}
\label{subsec:proof_of_lemma:symmetric_comonoidal_isomorphisms}
We show that both of the  maps:
\[
\begin{array}{lll}
  \jinv{R,S} := \func{J}R \oplus \func{J}S \mto^{\eta} \func{JH}(\func{J}R \oplus \func{J}S) \mto^{\func{J}\h{A,B}} \func{J}(\func{HJ}R + \func{HJ}S) \mto^{\J(\varepsilon_R + \varepsilon_S)} \func{J}(R + S)\\
  \\
  \jinv{0} := \perp \mto^{\eta} \func{JH}\perp \mto^{\func{J}\h{\perp}} \func{J}0
\end{array}
\]
are mutual inverses with $\j{R,S} : \func{J}(R + S) \mto \func{J}R
\oplus \func{J}S$ and $\j{0} : \perp \mto \func{J}0$ respectively.

\begin{itemize}
\item[Case.] The following diagram implies that $\jinv{R,S};\j{R,S} = \id$:
  \begin{diagram}
    \square|ammm|/->`->`->`<-/<950,500>[
      \func{J}R \oplus \func{J}S`
      \func{JH}(\func{J}R \oplus \func{J}S)`
      \func{JHJ}R \oplus \func{JHJ}S`
      \func{J}(\func{HJ}R + \func{HJ}S);
      \eta`
      \eta \oplus \eta`
      \func{J}\h{}`
      \j{}
    ]
    \dtriangle(-950,0)|amm|/=``<-/<950,500>[
      \func{J}R \oplus \func{J}S`
      \func{J}R \oplus \func{J}S`
      \func{JHJ}R \oplus \func{JHJ}S;``
      \func{J}\varepsilon \oplus \func{J}\varepsilon]

    \qtriangle(-950,-500)/`<-`->/<1900,500>[
      \func{J}R \oplus \func{J}S`
      \func{J}(\func{HJ}R + \func{HJ}S)`
      \func{J}(R + S);`
      \j{}`
      \func{J}(\varepsilon + \varepsilon)]        
  \end{diagram}
  The two top diagrams both commute because $\eta$ and $\varepsilon$
  are the unit and counit of the adjunction respectively, and the
  bottom diagram commutes by naturality of $\j{}$.
  
\item[Case.] The following diagram implies that $\j{R,S};\jinv{R,S} = \id$:
  \begin{diagram}
    \square|ammm|/->`->`->`->/<950,500>[
      \func{J}(R + S)`
      \func{J}R \oplus \func{J}S`
      \func{JHJ}(R + S)`
      \func{JH}(\func{J}R \oplus \func{J}S);
      \j{}`
      \eta`
      \eta`
      \func{JH}\j{}
    ]
    \dtriangle(-950,0)|amm|/=``<-/<950,500>[
      \func{J}(R + S)`
      \func{J}(R + S)`
      \func{JHJ}(R + S);``
      \func{J}\varepsilon]

    \qtriangle(-950,-500)/`<-`->/<1900,500>[
      \func{J}(R + S)`
      \func{JH}(\func{J}R \oplus \func{J}S)`
      \func{J}(\func{HJ}R + \func{HJ}S);`
      \func{J}(\varepsilon + \varepsilon)`
      \func{J}\h{}]
  \end{diagram}
  The top left and bottom diagrams both commute because $\eta$ and $\varepsilon$
  are the unit and counit of the adjunction respectively, and the
  top right diagram commutes by naturality of $\eta$.
  
\item[Case.] The following diagram implies that $\jinv{0};\j{0} = \id$:
  \begin{diagram}
    \square|amma|/->`=`->`<-/<950,500>[
      \perp`
      \func{JH}\perp`
      \perp`
      \func{J}0;
      \eta``
      \func{J}\h{\perp}`
      \j{0}]
  \end{diagram}
  This diagram holds because $\eta$ is the unit of the adjunction.

\item[Case.] The following diagram implies that $\j{0};\jinv{0} = \id$:        
  \begin{diagram}
    \Atriangle|aaa|/->`->`<-/<950,500>[
      \func{JHJ}0`
      \func{J}0`
      \func{JH}\perp;
      \func{J}\varepsilon`
      \func{JH}\j{0}`
      \func{J}\h{\perp}]

    \Dtriangle|aaa|/=`->`/<950,500>[
      \func{J}0`
      \func{JHJ}0`
      \func{J}0;`
      \eta`]

    \square/->``->`/<1900,1000>[
      \func{J}0`
      \perp`
      \func{J}0`
      \func{JH}\perp;
      \j{0}``
      \eta`]
  \end{diagram}
  The top-left and bottom diagrams commute because $\eta$ and
  $\varepsilon$ are the unit and counit of the adjunction
  respectively, and the top-right digram commutes by naturality of
  $\eta$.
\end{itemize}
% subsection proof_of_lemma~\ref{lemma:symmetric_comonoidal_isomorphisms} (end)

\subsection{Proof of Lemma~\ref{lemma:symmetric_comonoidal_monad}}
\label{subsec:proof_of_lemma:symmetric_comonoidal_monad}
Since $\wn$ is the composition of two symmetric comonoidal functors we know it is also symmetric comonoidal, and hence, the following diagrams all hold:
\begin{mathpar}
  \bfig
  \vSquares|ammmmma|/->`->`->``->`->`->/[
    \wn ((A \oplus B) \oplus C)`
    \wn (A \oplus B) \oplus \wn C`
    \wn (A \oplus (B \oplus C))`
    (\wn A \oplus \wn B) \oplus \wn C`
    \wn A \oplus \wn (B \oplus C))`
    \wn A \oplus (\wn B \oplus \wn C);
    \r{A \oplus B,C}`
    \wn \alpha_{A,B,C}`
    \r{A,B} \oplus \id_{\wn C}``
    \r{A,B \oplus C}`
    \alpha_{\wn A,\wn B,\wn C}`
    \id_{\wn A} \oplus \r{B,C}]    
  \efig
\end{mathpar}
%    \and
\begin{mathpar}
  \bfig
  \square|amma|/->`->`->`->/<1000,500>[
    \wn (\perp \oplus A)`
    \wn \perp \oplus \wn A`
    \wn A`
    \perp \oplus \wn A;
    \r{\perp,A}`
    \wn {\lambda}_{A}`
    \r{\perp} \oplus \id_{\wn A}`
      {\lambda^{-1}}_{\wn A}]
  \efig
  \and
  \bfig
  \square|amma|/->`->`->`->/<1000,500>[
    \wn (A \oplus \perp)`
    \wn A \oplus \wn \perp`
    \wn A`
    \wn A \oplus \perp;
    \r{A,\perp}`
    \wn {\rho}_{A}`
    \id_{\wn A} \oplus \r{\perp}`
       {\rho^{-1}}_{\wn A}]
  \efig
\end{mathpar}

\begin{diagram}
  \square|amma|/->`->`->`->/<1000,500>[
    \wn (A \oplus B)`
    \wn A \oplus \wn B`
    \wn (B \oplus A)`
    \wn B \oplus \wn A;
    \r{A,B}`
    \wn {\beta}_{A,B}`
        {\beta}_{\wn A,\wn B}`
        \r{B,A}]
\end{diagram}
Next we show that $(\wn,\eta,\mu)$ defines a monad where
$\eta_A : A \mto ?A$ is the unit of the adjunction, and
$\mu_A = \func{J}\varepsilon_{\func{H}\,A} : \wn\wn A \mto \wn A$.  It
suffices to show that every diagram of
Definition~\ref{def:symm-comonoidal-monad} holds.
\begin{itemize}
\item[Case.]
  $$\bfig
  \square|ammb|<600,600>[
    \wn^3 A`
    \wn^2 A`
    \wn^2 A`
    \wn A;
    \mu_{\wn A}`
    \wn\mu_A`
    \mu_A`
    \mu_A]
  \efig$$
  It suffices to show that the following diagram commutes:
  $$\bfig
  \square|ammb|<600,600>[
    \func{J}(\func{H}(\wn^2 A))`
    \func{J}(\func{H}\,\wn A)`
    \func{J}(\func{H}\,\wn A)`
    \func{J}(\func{H}\,A);
    \func{J}\varepsilon_{\func{H}\,\wn A}`
    \func{J}(\func{H}\,\mu_A)`
    \func{J}\varepsilon_{\func{H}\,A}`
    \func{J}\varepsilon_{\func{H}\,A}]
  \efig$$
  But this diagram is equivalent to the following:
  $$\bfig
  \square|ammb|<600,600>[
    \func{H}\func{JHJH} A`
    \func{H}\,\func{JH} A`
    \func{H}\,\func{JH} A`
    \func{H}\,A;
    \varepsilon_{\func{H}\,\func{JH} A}`
    \func{H}\,\func{J}\varepsilon_{\func{H}\,A}`
    \varepsilon_{\func{H}\,A}`
    \varepsilon_{\func{H}\,A}]
  \efig$$
  The previous diagram commutes by naturality of $\varepsilon$.

\item[Case.]
  $$\bfig
  \Atrianglepair/=`<-`=`->`<-/<600,600>[
    \wn A`
    \wn A`
    \wn^2 A`
    \wn A;`
    \mu_A``
    \eta_{\wn A}`
    \wn \eta_A]
  \efig$$
  It suffices to show that the following diagrams commutes:
  $$\bfig
  \Atrianglepair/=`<-`=`->`<-/<600,600>[
    JH A`
    JH A`
    JHJH A`
    JH A;`
    J\varepsilon_{HA}``
    \eta_{JH A}`
    JH \eta_A]
  \efig$$
  Both of these diagrams commute because $\eta$ and $\varepsilon$
  are the unit and counit of an adjunction.
\end{itemize}

It remains to be shown that $\eta$ and $\mu$ are both
symmetric comonoidal natural transformations, but this easily follows
from the fact that we know $\eta$ is by assumption, and that $\mu$
is because it is defined in terms of $\varepsilon$ which is a
symmetric comonoidal natural transformation.  Thus, all of the
following diagrams commute:
\begin{mathpar}
  \bfig
  \ptriangle|amm|/->`->`<-/<1000,600>[
    A \oplus B`
    \wn A \oplus \wn B`
    \wn (A \oplus B);
    \eta_A \oplus \eta_B`
    \eta_A`
    \r{A,B}]    
  \efig
  \and
  \bfig
  \Vtriangle/->`=`->/<600,600>[
    \perp`
    \wn\perp`
    \perp;
    \eta_\perp``
    \r{\perp}]
  \efig
\end{mathpar}
\begin{mathpar}
  \bfig
  \square|ammm|/->`->``/<1050,600>[
    \wn^2(A \oplus B)`
    \wn (\wn A \oplus \wn B)`
    \wn (A \oplus B)`;
    \wn\r{A,B}`
    \mu_{A \oplus B}``]

  \square(850,0)|ammm|/->``->`/<1050,600>[
    \wn (\wn A \oplus \wn B)`
    \wn^2 A \oplus \wn^2 B``
    \wn A \oplus \wn B;
    \r{\wn A,\wn B}``
    \mu_A \oplus \mu_B`]
  \morphism(-200,0)<2100,0>[\wn(A \oplus B)`\wn A \oplus \wn B;\r{A,B}]
  \efig
  \and
  \bfig
  \square|ammb|/->`->`->`->/<600,600>[
    \wn^2\perp`
    \wn\perp`
    \wn\perp`
    \perp;
    \wn\r{\perp}`
    \mu_\perp`
    \r{\perp}`
    \r{\perp}]
  \efig
\end{mathpar}
% subsection proof_of_lemma:symmetric_monoidal_monad (end)

\subsection{Proof of Lemma~\ref{lemma:right_weakening_and_contraction}}
\label{subsec:proof_of_lemma:right_weakening_and_contraction}
Suppose $(\func{H},\h{})$ and $(\func{J},\j{})$ are two symmetric
comonoidal functors, such that, $\cat{L} : \func{H} \dashv \func{J}
: \cat{C}$ is a coLNL model.  Again, we know $\wn A = H;J : \cat{L}
\mto \cat{L}$ is a symmetric comonoidal monad by
Lemma~\ref{lemma:symmetric_comonoidal_monad}.  

We define the following morphisms:
\[
\begin{array}{lll}
  \w{A} := \perp \mto^{\jinv{0}} \func{J} 0 \mto^{\func{J}\diamond_{\func{H} A}} \func{JH}A \mto/=/ \wn A\\
  \c{A} := \wn A \oplus \wn A \mto/=/ \func{JH}A \oplus \func{JH}A \mto^{\jinv{\func{H}A,\func{H}A}} \func{J}(\func{H}A + \func{H}A) \mto^{\func{J}\codiag{\func{H}A}} \func{JHA} \mto/=/ \wn A
\end{array}
\]

Next we show that both of these are symmetric comonoidal natural
transformations, but for which functors?  Define $\func{W}(A) =
\perp$ and $\func{C}(A) = \wn A \oplus \wn A$ on objects of
$\cat{L}$, and $\func{W}(f : A \mto B) = \id_\perp$ and $\func{C}(f
: A \mto B) = \wn f \oplus \wn f$ on morphisms.  So we must show
that $\w{} : \func{W} \mto \wn$ and $\c{} : \func{C} \mto \wn$ are
symmetric comonoidal natural transformations.  We first show that
$\w{}$ is and then we show that $\c{}$ is.  Throughout the proof we
drop subscripts on natural transformations for readability.
\begin{itemize}
\item[Case.] To show $\w{}$ is a natural transformation we must show
  the following diagram commutes for any morphism $f : A \mto B$:
  \[
  \bfig
  \square[W(A)`\wn A`W(B)`\wn B;\w{A}`W(f)`\wn f`\w{B}]
  \efig
  \]
  This diagram is equivalent to the following:
  \[
  \bfig
  \square[\perp`\wn A`\perp`\wn B;\w{A}`\id_{\perp}`\wn f`\w{B}]
  \efig
  \]
  It further expands to the following:
  \[
  \bfig
  \hSquares/->`->`->``->`->`->/[\perp`\func{J}0`\func{JH}A`\perp`\func{J}0`\func{JH}B;\jinv{0}`\func{J}(\diamond_{\func{H}A})`\id_\perp``\func{JH}f`\jinv{0}`\func{J}(\diamond_{\func{H}B})]
  \efig
  \]
  This diagram commutes, because
  $\func{J}(\diamond_{\func{H}A});\func{J}f =
  \func{J}(\diamond_{\func{H}A};f) =
  \func{J}(\diamond_{\func{H}B})$, by the uniqueness of the initial
  map.

\item[Case.] The functor $\func{W}$ is comonoidal itself.  To see this we
  must exhibit a map
  \[\s{\perp} := \id_\perp : \func{W}\perp \mto \perp\]
  and a natural transformation
  \[\s{A,B} := \rho^{-1}_\perp : \func{W}(A \oplus B) \mto \func{W}A \oplus \func{W}B\]
  subject to the coherence conditions in
  Definition~\ref{def:coSMCFUN}.  Clearly, the second map is a natural
  transformation, but we leave showing they respect the coherence
  conditions to the reader.  Now we can show that $\w{}$ is indeed
  symmetric comonoidal.
  \begin{itemize}
  \item[Case.] \ \\
    \begin{diagram}
      \square|amma|<1000,500>[
        \func{W}(A \oplus B)`
        \func{W}A \oplus \func{W}B`
        \wn (A \oplus B)`
        \wn A \oplus \wn B;
        \s{A,B}`
        \w{A \oplus B}`
        \w{A} \oplus \w{B}`
        \r{A,B}]
    \end{diagram}
    Expanding the objects of the previous diagram results in the
    following:
    \begin{diagram}
      \square|amma|<1000,500>[
        \perp`
        \perp \oplus \perp`
        \wn (A \oplus B)`
        \wn A \oplus \wn B;
        \s{A,B}`
        \w{A \oplus B}`
        \w{A} \oplus \w{B}`
        \r{A,B}]
    \end{diagram}
    This diagram commutes, because the following fully expanded
    diagram commutes:
    \begin{diagram}
      \square|amma|/<-`->`->`->/<950,500>[
        \J 0`
        \J (0 + 0)`
        \J\H (A \oplus B)`
        \J (\H A + \H B);
        \J\rho`
        \J\diamond`
        \J (\diamond + \diamond)`
        \J\h{}]

      \square(950, 0)|amma|/->``->`->/<950,500>[
        \J (0 + 0)`
        \J 0 + \J 0`
        \J (\H A + \H B)`
        \J\H A \oplus \J\H B;
        \j{}``
        \J\diamond \oplus \J\diamond`
        \j{}]

      \square(0,500)/->`->``/<1900,1500>[
        \perp`
        \perp \oplus \perp`
        \J 0`;
        \rho^{-1}`
        \jinv{0}``]

      \dtriangle(950,1300)|mma|<950,700>[
        \perp \oplus \perp`
        \J 0 \oplus \perp`
        \J 0 \oplus \J 0;
        \jinv{0} \oplus \id`
        \jinv{0} \oplus \jinv{0}`
        \id \oplus \jinv{0}]

      \ptriangle(950,800)|amm|/`=`->/<950,500>[
        \J 0 \oplus \perp`
        \J 0 \oplus \J 0`
        \J 0 \oplus \perp;``
        \id \oplus \j{0}]

      \morphism(1900,1300)|m|/=/<0,-800>[
        \J 0 \oplus \J 0`
        \J 0 \oplus \J 0;]

      \morphism(0,500)|m|<950,800>[
        \J 0`
        \J0 \oplus \perp;
        \rho^{-1}]

      \place(475,250)[(1)]
      \place(1425,250)[(2)]
      \place(950,650)[(3)]
      \place(1180,1100)[(4)]
      \place(1620,1550)[(5)]
      \place(475,1550)[(6)]
    \end{diagram}
    Diagram 1 commutes because $0$ is the initial object, diagram 2
    commutes by naturality of $\j{}$, diagram 3 commutes because
    $\J$ is a symmetric comonoidal functor, diagram 4 commutes
    because $\j{0}$ is an isomorphism
    (Lemma~\ref{lemma:symmetric_comonoidal_isomorphisms}), diagram 5
    commutes by functorality of $\J$, and diagram 6 commutes by
    naturality of $\rho$.
    
  \item[Case.]\ \\
    \begin{diagram}
      \Vtriangle/<-`<-`<-/[
        \perp`
        \wn \perp`
        \func{W}\perp;
        \r{\perp}`
        \s{\perp}`
        \w{\perp}]
    \end{diagram}
    Expanding the objects in the previous diagram results in the
    following:
    \begin{diagram}
      \Vtriangle/<-`=`<-/[
        \perp`
        \wn \perp`
        \perp;
        \r{\perp}``
        \w{\perp}]
    \end{diagram}
    This diagram commutes because the following one does:
    \begin{diagram}
      \dtriangle|ama|/=`<-`->/<950,500>[
        \J 0`
        \J 0`
        \J\H \perp;`
        \J\h{\perp}`
        \J\diamond]
      \square(-950,0)|aaaa|/`=``->/<950,500>[
        \perp``
        \perp`
        \J 0;```
        \jinv{0}]
      \morphism(-950,500)/<-/<1900,0>[
        \perp`
        \J 0;
        \j{0}]        
    \end{diagram}
    The diagram on the left commutes because $\j{0}$ is an
    isomorphism
    (Lemma~\ref{lemma:symmetric_comonoidal_isomorphisms}), and the
    diagram on the right commutes because $0$ is the initial object.

  \end{itemize}

\item[Case.] Now we show that $\c{A} : \wn A \oplus \wn A \mto \wn
  A$ is a natural transformation.  This requires the following
  diagram to commute (for any $f : A \mto B$):
  \[
  \bfig
  \square[
    \func{C}A`
    \wn A`
    \func{C}B`
    \wn B;
    \c{A}`
    \func{C}f`
    \wn f`
    \c{B}]
  \efig
  \]
  This expands to the following diagram:
  \[
  \bfig
  \square[
    \wn A \oplus \wn A`
    \wn A`
    \wn B \oplus \wn B`
    \wn B;
    \c{A}`
    \wn f \oplus \wn f`
    \wn f`
    \c{B}]
  \efig
  \]
  This diagram commutes because the following diagram does:
  \begin{diagram}
    \hSquares[
      \func{JH}A \oplus \func{JH}A`
      \func{J}(\func{H}A + \func{H}A)`
      \func{JH}A`
      \func{JH}B \oplus \func{JH}B`
      \func{J}(\func{H}B + \func{H}B)`
      \func{JH}B;
      \jinv{\func{H}A, \func{H}A}`
      \func{J}\bigtriangledown_{\func{H}A}`
      \func{JH}f \oplus \func{JH}f`
      \func{J}(\func{H}f + \func{H}f)`
      \func{JH}f`
      \jinv{\func{H}B, \func{H}B}`
      \func{J}\bigtriangledown_{\func{H}B}]
  \end{diagram}
  The left square commutes by naturality of $\jinv{}$, and the right square commutes by naturality of the codiagonal
  $\bigtriangledown_{A} : A + A \mto A$.

\item[Case.] The functor $\func{C} : \cat{L} \mto \cat{L}$ is indeed
  symmetric comonoidal where the required maps are defined as follows:
  \[
  \small
  \begin{array}{lll}
    \quad\quad\quad\t_\perp := \wn \perp \oplus \wn \perp \mto/=/ \J\H\perp \oplus \J\H\perp \mto^{\jinv{}} \J(\H\perp + \H\perp) \mto^{\J\codiag{}} \J\H\perp \mto^{\J\h{\perp}} \J 0 \mto^{\j{0}} \perp\\
    \\
    \quad\quad\quad\t_{A,B} := \wn (A \oplus B) \oplus \wn (A \oplus B) \mto^{\r{A,B} \oplus \r{A,B}} (\wn A \oplus \wn B) \oplus (\wn A \oplus \wn B) \mto^{\iso} (\wn A \oplus \wn A) \oplus (\wn B \oplus \wn B)
  \end{array}
  \]
  where $\mathsf{iso}$ is a natural isomorphism that can easily be
  defined using the symmetric monoidal structure of
  $\cat{L}$. Clearly, $\t$ is indeed a natural transformation, but
  we leave checking that the required diagrams in
  Definition~\ref{def:coSMCFUN} commute to the reader.  We can now
  show that $\c{A} : \wn A \oplus \wn A \mto \wn A$ is symmetric
  comonoidal.  The following diagrams from
  Definition~\ref{def:coSMCNAT} must commute:
  \begin{itemize}
  \item[Case.] \ \\
    \begin{diagram}
      \square<1000,500>[
        \func{C}(A \oplus B)`
        \func{C}A \oplus \func{C}B`
        \wn (A \oplus B)`
        \wn A \oplus \wn B;
        \t_{A,B}`
        \c{A \oplus B}`
        \c{A} \oplus \c{B}`
        \r{A,B}]
    \end{diagram}
    Expanding the objects in the previous diagram results in the following:
    \begin{diagram}
      \square<2000,500>[
        \wn (A \oplus B) \oplus \wn (A \oplus B)`
        (\wn A \oplus \wn A) \oplus (\wn B \oplus \wn B)`
        \wn (A \oplus B)`
        \wn A \oplus \wn B;
        \t_{A,B}`
        \c{A \oplus B}`
        \c{A} \oplus \c{B}`
        \r{A,B}]
    \end{diagram}
    This diagram commutes, because the following fully expanded one
    does:
    \begin{center}
      \rotatebox{90}{$
        \bfig
        \square|amma|<1500,500>[
          \J(\H (A \oplus B) + \H (A \oplus B))`
          \J((\H A + \H B) + (\H A + \H B))`
          \J\H (A \oplus B)`
          \J (\H A + \H B);
          \J(\h{} + \h{})`
          \J\codiag{}`
          \J\codiag{}`
          \J\h{}]

        \square(0,500)|amma|<1500,500>[
          \J\H(A \oplus B) \oplus \J\H(A \oplus B)`
          \J(\H A + \H B) \oplus \J(\H A + \H B)`
          \J(\H(A \oplus B) + \H(A \oplus B))`
          \J((\H A + \H B) + (\H A + \H B));
          \J\h{} \oplus \J\h{}`
          \jinv{}`
          \jinv{}`
          \J(\h{} + \h{})]

        \square(1500,0)|amma|/->`->`->`=/<1500,500>[
          \J((\H A + \H B) + (\H A + \H B))`
          \J((\H A + \H A) + (\H B + \H B))`
          \J (\H A + \H B)`
          \J (\H A + \H B);
          \J\iso`
          \J\codiag{}`
          \J(\codiag{} + \codiag{})`]

        \square(3000,0)|amma|<1500,500>[
          \J((\H A + \H A) + (\H B + \H B))`
          \J(\H A + \H A) \oplus \J(\H B + \H B)`
          \J (\H A + \H B)`
          \J\H A \oplus \J\H B;
          \j{}`
          \J(\codiag{} + \codiag{})`
          \J\codiag{} \oplus \J\codiag{}`
          \j{}]

        \dtriangle(3000,500)|ama|/<-`->`->/<1500,500>[
          (\J\H A \oplus \J\H A) \oplus (\J\H B \oplus \J\H B)`
          \J((\H A + \H A) + (\H B + \H B))`
          \J(\H A + \H A) \oplus \J(\H B + \H B);
          \j{};(\j{} \oplus \j{})`
          \jinv{} \oplus \jinv{}`
          \j{}]          

        \morphism(1500,1000)<1500,0>[
          \J(\H A + \H B) \oplus \J(\H A + \H B)`
          (\J\H A \oplus \J\H B) \oplus (\J\H A \oplus \J\H B);
          \j{} \oplus \j{}]

        \morphism(3000,1000)<1500,0>[
          (\J\H A \oplus \J\H B) \oplus (\J\H A \oplus \J\H B)`
          (\J\H A \oplus \J\H A) \oplus (\J\H B \oplus \J\H B);
          \iso]

        \place(750,250)[(1)]
        \place(750,750)[(2)]
        \place(2250,250)[(3)]
        \place(2550,750)[(4)]
        \place(3750,250)[(5)]
        \place(4125,700)[(6)]
        \efig
        $}
    \end{center}
    Diagram 1 commutes by naturality of $\codiag{}$, diagram 2
    commutes by naturality of $\jinv{}$, diagram 3 commutes by
    straightforward reasoning on coproducts, diagram 4 commutes by
    straightforward reasoning on the symmetric monoidal structure of
    $\J$ after expanding the definition of the two isomorphisms --
    here $\J\iso$ is the corresponding isomorphisms on coproducts --
    diagram 5 commutes by naturality of $\j{}$, and diagram 6
    commutes because $\j{}$ is an isomorphism
    (Lemma~\ref{lemma:symmetric_comonoidal_isomorphisms}).
    
  \item[Case.] \ \\
    \begin{diagram}
      \Vtriangle/<-`<-`<-/[
        \perp`
        \wn \perp`
        \func{C} \perp;
        \r{\perp}`
        \t{\perp}`
        \c{\perp}]
    \end{diagram}
    Expanding the objects of this diagram results in the following:
    \begin{diagram}
      \square/<-`<-`<-`=/<950,500>[
        \perp`
        \wn \perp`
        \wn \perp \oplus \wn \perp`
        \wn \perp \oplus \wn \perp;
        \r{\perp}`
        \t{\perp}`
        \c{\perp}`]          
    \end{diagram}
    Simply unfolding the morphisms in the previous diagram reveals the following:
    \begin{diagram}
      \square/`<-`<-`=/<950,500>[
        \J(\H\perp + \H\perp)`
        \J(\H\perp + \H\perp)`
        \J\H\perp \oplus \J\H\perp`
        \J\H\perp \oplus \J\H\perp;`
        \jinv{}`
        \jinv{}`]

      \square(0,500)/`<-`<-`/<950,500>[
        \J\H\perp`
        \J\H\perp`
        \J(\H\perp + \H\perp)`
        \J(\H\perp + \H\perp);
        `
        \J\codiag{}`
        \J\codiag{}`]

      \square(0,1000)/`<-`<-`/<950,500>[
        \J0`
        \J0`
        \J\H\perp`
        \J\H\perp;
        `
        \J\h{\perp}`
        \J\h{\perp}`]

      \square(0,1500)/=`<-`<-`/<950,500>[
        \perp`
        \perp`
        \J0`
        \J0;
        `
        \j{}`
        \j{}`]
    \end{diagram}
    Clearly, this diagram commutes.
  \end{itemize}
\end{itemize}
At this point we have shown that $\w{A} : \perp \mto \wn A$ and
$\c{A} : \wn A \oplus \wn A \mto \wn A$ are symmetric comonoidal
naturality transformations.  Now we show that for any $\wn A$ the
triple $(\wn A,\w{A},\c{A})$ forms a commutative monoid.  This means
that the following diagrams must commute:
\begin{itemize}
\item[Case.]\ \\
  \begin{diagram}
    \hSquares|aammmmm|/->`->```->``/[
      (\wn A \oplus \wn A) \oplus \wn A`
      \wn A \oplus (\wn A \oplus \wn A)`
      \wn A \oplus \wn A```
      \wn A;
      \alpha_{\wn A,\wn A,\wn A}`
      \id_{\wn A} \oplus \c{A}```
      \c{A}``]
    \btriangle|maa|/->``->/<2407,500>[(\wn A \oplus \wn A) \oplus \wn A`
      \wn A \oplus \wn A`
      \wn A;
      \c{A} \oplus \id_{A}``
      \c{A}]
  \end{diagram}
  The previous diagram commutes, because the following one does (we
  omit subscripts for readability):
  \begin{diagram}
    \scriptsize
    \square|amma|/->`->``->/<950,500>[
      (\func{JH}A \oplus \func{JH}A) \oplus \func{JH}A`
      \func{JH}A \oplus (\func{JH}A \oplus \func{JH}A)`
      \func{J}(\func{H}A + \func{H}A) \oplus \func{JH}A`
      \func{J}((\func{H}A + \func{H}A) + \func{H}A);
      \alpha`
      \jinv{} \oplus \id``
      \jinv{}]

    \square(950,0)|amma|/->``->`->/<950,500>[
      \func{JH}A \oplus (\func{JH}A \oplus \func{JH}A)`
      \func{JH}A \oplus \func{J}(\func{H}A + \func{H}A)`
      \func{J}((\func{H}A + \func{H}A) + \func{H}A)`
      \func{J}(\func{H}A + (\func{H}A + \func{H}A));
      \id \oplus \jinv{}``
      \jinv{}`
      \func{J}\alpha]

    \square(1900,0)|amma|/->``->`->/<950,500>[
      \func{JH}A \oplus \func{J}(\func{H}A + \func{H}A)`
      \func{JH}A \oplus \func{JH}A`
      \func{J}(\func{H}A + (\func{H}A + \func{H}A))`
      \func{J}(\func{H}A + \func{H}A);
      \id \oplus \func{J}\codiag{}``
      \jinv{}`
      \func{J}(\id + \codiag{})]

    \square(0,-500)|amma|/`->`->`->/<950,500>[
      \func{J}(\func{H}A + \func{H}A) \oplus \func{JH}A`
      \func{J}((\func{H}A + \func{H}A) + \func{H}A)`
      \func{JH}A \oplus \func{JH}A`
      \func{J}(\func{H}A + \func{H}A);`
      \func{J}\codiag{} \oplus \id`
      \func{J}(\codiag{} + \id)`
      \jinv{}]

    \dtriangle(950,-500)|ama|/`->`->/<1900,500>[
      \func{J}(\func{H}A + \func{H}A)`
      \func{J}(\func{H}A + \func{H}A)`
      \func{JH}A;`
      \func{J}\codiag{}`
      \func{J}\codiag{}]

    \place(950,250)[(1)]
    \place(2375,250)[(2)]
    \place(475,-250)[(3)]
    \place(1900,-250)[(4)]
  \end{diagram}
  Diagram 1 commutes because $\func{J}$ is a symmetric monoidal
  functor (Corollary~\ref{corollary:J-SMMF}), diagrams 2 and 3
  commute by naturality of $\jinv{}$, and diagram 4 commutes because
  $(\func{H}A, \diamond, \codiag{})$ is a commutative monoid in
  $\cat{C}$, but we leave the proof of this to the reader.

\item[Case.]\ \\
  \begin{diagram}
    \btriangle|maa|/->`->`->/<1000,600>[
      \wn A \oplus \perp`
      \wn A \oplus \wn A`
      \wn A;
      \id_{\wn A} \oplus \w{A}`
      \rho_{\wn A}`
      \c{A}]
  \end{diagram}
  The previous diagram commutes, because the following one does:
  \begin{diagram}
    \square|amma|/->`->`->`->/<950,500>[
      \func{JH}A \oplus \func{J}0`
      \func{J}(\func{H}A + 0)`
      \func{JH}A \oplus \func{JH}A`
      \func{J}(\func{H}A + \func{H}A);
      \jinv{}`
      \id \oplus \func{J}\diamond`
      \func{J}(\id \oplus \diamond)`
      \jinv{}]

    \square(950,0)|amma|/->``=`->/<950,500>[
      \func{J}(\func{H}A + 0)`
      \func{JH}A`
      \func{J}(\func{H}A + \func{H}A)`
      \func{JH}A;
      \func{J}\rho```
      \func{J}\codiag{}]

    \square(0,500)|amma|/->`->`=`/<1900,500>[
      \func{JH}A \oplus \perp`
      \func{JH}A`
      \func{JH}A \oplus \func{J}0`
      \func{JH}A;
      \rho`
      \id \oplus \jinv{0}``]

    \place(950,750)[(1)]
    \place(475,250)[(2)]
    \place(1425,250)[(3)]      
  \end{diagram}
  Diagram 1 commutes because $\func{J}$ is a symmetric monoidal
  functor (Corollary~\ref{corollary:J-SMMF}), diagram 2 commutes by
  naturality of $\jinv{}$, and diagram 3 commutes because
  $(\func{H}A, \diamond, \codiag{})$ is a commutative monoid in
  $\cat{C}$, but we leave the proof of this to the reader.
  
\item[Case.]\ \\
  \begin{diagram}
    \btriangle|maa|/->`->`->/<1000,600>[
      \wn A \oplus \wn A`
      \wn A \oplus \wn A`
      \wn A;
      \beta_{\wn A,\wn A}`
      \c{A}`
      \c{A}]
  \end{diagram}
  This diagram commutes, because the following one does:
  \begin{diagram}
    \hSquares/->`->`->`->`=`->`->/[
      \func{JH}A \oplus \func{JH}A`
      \func{J}(\func{H}A + \func{H}A)`
      \func{JH}A`
      \func{JH}A \oplus \func{JH}A`
      \func{J}(\func{H}A + \func{H}A)`
      \func{JH}A;
      \jinv{}`
      \func{J}\codiag{}`
      \beta`
      \func{J}\beta``
      \jinv{}`
      \func{J}\codiag{}]
  \end{diagram}
  The left diagram commutes by naturality of $\jinv{}$, and the right
  diagram commutes because $(\func{H}A, \diamond, \codiag{})$ is a
  commutative monoid in $\cat{C}$, but we leave the proof of this to
  the reader.
\end{itemize}

Finally, we must show that $\w{A} : \perp \mto \wn A$ and $\c{A} :
\wn A \oplus \wn A \mto \wn A$ are $\wn\text{-algebra}$ morphisms.
The algebras in play here are $(\wn A,\mu : \wn\wn A \mto \wn A)$,
$(\perp, \r{\perp} : \wn \perp \mto \perp)$, and $(\wn A \oplus \wn
A, u_A : \wn (\wn A \oplus \wn A) \mto \wn A \oplus \wn A)$, where
$u_A := \wn (\wn A \oplus \wn A) \mto^{\r{\wn A,\wn A}} ?^2 A \oplus
?^2 A \mto^{\mu_A \oplus \mu_A} \wn A \oplus \wn A$.  It suffices to
show that the following diagrams commute:
\begin{itemize}
\item[Case.]\ \\
  \begin{diagram}
    \square<950,500>[
      \wn \perp`
      \perp`
      \wn\wn A`
      \wn A;
      \r{\perp}`
      \wn\w{}`
      \w{}`
      \mu]
  \end{diagram}
  This diagram commutes, because the following fully expanded one does:
  \begin{diagram}
    \square|mmma|<2500,500>[
      \J\H\J 0`
      \J 0`
      \J\H\J\H A`
      \J\H A;
      \J\varepsilon_0`
      \J\H\J\diamond`
      \J\diamond`
      \J\varepsilon]

    \square(0,500)|amma|/->`->``/<1250,1000>[
      \J\H\perp`
      \J 0`
      \J\H\J 0`;
      \J\h{\perp}`
      \J\H\jinv{0}``]
    \square(1250,500)|amma|/->``->`/<1250,1000>[
      \J 0`
      \perp``
      \J 0;
      \j{0}``
      \jinv{0}`]

    \Dtriangle(0,500)|mmm|/`->`=/<1250,500>[
      \J\H \perp`
      \J\H\J 0`
      \J\H\J 0;`
      \J\H\jinv{0}`]

    \morphism(1250,1000)|m|<700,0>[
      \J\H\J 0`
      \J\H \perp;
      \J\H\j{0}]

    \morphism(1950,1000)|m|<550,-500>[
      \J\H \perp`
      \J 0;
      \J\h{\perp}]

    \place(1250,250)[(1)]
    \place(1500,750)[(2)]
    \place(380,1000)[(3)]
    \place(2000,1250)[(4)]
  \end{diagram}
  Diagram 1 commutes by naturality of $\varepsilon$, diagram 2
  commutes because $\varepsilon$ is the counit of the symmetric
  comonoidal adjunction, diagram 3 clearly commutes, and diagram 4
  commutes because $\j{0}$ is an isomorphism
  (Lemma~\ref{lemma:symmetric_comonoidal_isomorphisms}).
  
\item[Case.]\ \\
  \begin{diagram}
    \square|amma|<950,500>[
      \wn (\wn A \oplus \wn A)`
      \wn A \oplus \wn A`
      \wn\wn A`
      \wn A;
      u`
      \wn\c{}`
      \c{}`
      \mu]
  \end{diagram}
  This diagram commutes because the following fully expanded one does:
  \begin{center}
    \tiny
    \rotatebox{90}{$\bfig
      \square|amma|/->`->``/<1250,1500>[
        \J\H\J(\H A + \H A)`
        \J\H(\J\H A \oplus \J\H A)`
        \J\H\J\H A`;
        \J\H\j{}`
        \J\H\J\codiag{}``]
      \square(1250,0)|amma|/->``->`/<1250,1500>[
        \J\H(\J\H A \oplus \J\H A)`
        \J(\H\J\H A + \H\J\H A)``
        \J\H\J\H A;
        \J\h{}``
        \J\codiag{}`]
      \morphism/=/<2500,0>[\J\H\J\H A`\J\H\J\H A;]

      \Atriangle/<-`<-`=/<1250,500>[
        \J\H A`
        \J\H\J\H A`
        \J\H\J\H A;
        \J\varepsilon`
        \J\varepsilon`]

      \Vtriangle(0,1000)/`->`->/<1250,500>[
        \J\H\J(\H A + \H A)`
        \J(\H\J\H A + \H\J\H A)`
        \J(\H A + \H A);`
        \J\varepsilon`
        \J(\varepsilon + \varepsilon)]

      \morphism(1250,1000)|m|<0,-500>[
        \J(\H A + \H A)`
        \J\H A;
        \J\codiag{}]

      \square(0,1500)|amma|/->`->``/<1250,500>[
        \J\H(\J\H A \oplus \J\H A)`
        \J(\H\J\H A + \H\J\H A)`
        \J\H\J(\H A + \H A)`;
        \J\h{}`
        \J\H\jinv{}``]        

      \square(1250,1500)|amma|/->``->`/<1250,500>[
        \J(\H\J\H A + \H\J\H A)`
        \J\H\J\H A \oplus \J\H\J\H A`
        \J\H(\J\H A \oplus \J\H A)`
        \J(\H\J\H A + \H\J\H A);
        \j{}`
        `
        \jinv{}`]

      \square(2500,0)|amma|<1250,1500>[
        \J(\H\J\H A + \H\J\H A)`
        \J(\H A + \H A)`
        \J\H\J\H A`
        \J\H A;
        \J(\varepsilon + \varepsilon)`
        \J\codiag{}`
        \J\codiag{}`
        \J\varepsilon]

      \square(2500,1500)|amma|<1250,500>[
        \J\H\J\H A \oplus \J\H\J\H A`
        \J\H A \oplus \J\H A`
        \J(\H\J\H A + \H\J\H A)`
        \J(\H A + \H A);
        \J\varepsilon \oplus \J\varepsilon`
        \jinv{}`
        \jinv{}`
        \J(\varepsilon + \varepsilon)]

      \place(1250,250)[(1)]
      \place(625,750)[(2)]
      \place(1875,750)[(3)]
      \place(1250,1250)[(4)]
      \place(1250,1750)[(5)]
      \place(3125,1750)[(6)]
      \place(3125,750)[(7)]
      \efig$}
  \end{center}
\end{itemize}
Diagram 1 clearly commutes, diagram 2 commutes by naturality of
$\varepsilon$, diagram 3 commutes by naturality of $\codiag{}$,
diagram 4 commutes because $\varepsilon$ is the counit of the
symmetric comonoidal adjunction, diagram 5 commutes because $\j{}$
is an isomorphism
(Lemma~\ref{lemma:symmetric_comonoidal_isomorphisms}), diagram 6
commutes by naturality of $\jinv{}$, and diagram 7 is the same
diagram as 3, but this diagram is redundant for readability.
% subsection proof_of_lemma:right_weakening_and_contraction (end)

\section{Proof of Lemma~\ref{lemma:monoid-morphism}}
\label{sec:proof_of_lemma:monoid-morphism}
Suppose $\cat{L} : \func{H} \dashv \func{J} : \cat{C}$ is a coLNL
model.  Then we know $\wn A = \J\H A$ is a symmetric comonoidal
monad by Lemma~\ref{lemma:symmetric_comonoidal_monad}.  Bellin
\cite{Bellin:2012} remarks that by Maietti, Maneggia de Paiva and
Ritter's Proposition~25 \cite{Maietti2005}, it suffices to show that
$\mu_A : \wn\wn A \mto \wn A$ is a monoid morphism.  Thus, the
following diagrams must commute:
\begin{itemize}
\item[Case.]\ \\
  \begin{diagram}
    \square|amma|<950,500>[
      \wn\wn A \oplus \wn\wn A`
      \wn\wn A`
      \wn A \oplus \wn A`
      \wn A;
      \c{\wn A}`
      \mu_A \oplus \mu_A`
      \mu_A`
      \c{A}]
  \end{diagram}
  This diagram commutes because the following fully expanded one
  does:
  \begin{diagram}
    \square|amma|<1000,500>[
      \J\H\J\H A \oplus \J\H\J\H A`
      \J(\H\J\H A + \H\J\H A)`
      \J\H A \oplus \J\H A`
      \J(\H A + \H A);
      \jinv{}`
      \J\varepsilon \oplus \J\varepsilon`
      \J(\varepsilon + \varepsilon)`
      \jinv{}]

    \square(1000,0)|amma|<1000,500>[
      \J(\H\J\H A + \H\J\H A)`
      \J\H\J\H A`
      \J(\H A + \H A)`
      \J\H A;
      \J\codiag{}`
      \J(\varepsilon + \varepsilon)`
      \J\varepsilon`
      \J\codiag{}]      
  \end{diagram}
  The left square commutes by naturality of $\jinv{}$ and the right
  square commutes by naturality of the codiagonal.
  
\item[Case.]\ \\
  \begin{diagram}
    \Atriangle|aaa|[
      \perp`
      \wn\wn A`
      \wn A;
      \w{\wn A}`
      \w{A}`
      \mu_A]
  \end{diagram}
  This diagram commutes because the following fully expanded one
  does:
  \begin{diagram}
    \square|amma|/=`->`->`=/<1000,500>[
      \perp`
      \perp`
      \J 0`
      \J 0;`
      \jinv{0}`
      \jinv{0}`]

    \square(0,-500)|amma|/=`->`->`->/<1000,500>[
      \J 0`
      \J 0`
      \J\H\J\H A`
      \J\H A;`
      \J\diamond`
      \J\diamond`
      \J\varepsilon]
  \end{diagram}
  The top square trivially commutes, and the bottom square commutes
  by uniqueness of the initial map.
\end{itemize}
% section proof_of_lemma:monoid-morphism (end)

% section proofs (end)

\end{document}

%%% Local Variables: 
%%% mode: latex
%%% TeX-master: t
%%% End: 

