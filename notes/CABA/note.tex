\documentclass{article}
\usepackage{parskip}
\usepackage{amsmath,amssymb,amsthm}
\usepackage{fullpage}
\usepackage{hyperref}
\usepackage{color}
\usepackage{tikz}
\usepackage{mathabx}
\usetikzlibrary{arrows,automata,positioning,decorations.pathreplacing}
\usepackage[barr]{xy}

\newtheorem{thm}{Theorem}
\newtheorem{lemma}[thm]{Lemma}
\newtheorem{corollary}[thm]{Corollary}
\newtheorem{definition}[thm]{Definition}
\newtheorem{remark}[thm]{Remark}
\newtheorem{proposition}[thm]{Proposition}
\newtheorem{notn}[thm]{Notation}
\newtheorem{observation}[thm]{Observation}
\newtheorem{example}[thm]{Example}

\newcommand{\powerset}[1]{\mathcal{P}(#1)}
\newcommand{\comp}[1]{\overline{#1}}
\newcommand{\CABAC}[0]{\mathsf{CABACat}}
\newcommand{\Set}[0]{\mathsf{Set}}
\newcommand{\inO}[1]{\mathsf{in}_1(#1)}
\newcommand{\inT}[1]{\mathsf{in}_2(#1)}
\newcommand{\imp}[0]{\Rightarrow}
\newcommand{\cur}[0]{\mathsf{cur}}
\newcommand{\cocur}[0]{\mathsf{cocur}}
\newcommand{\id}[0]{\mathsf{id}}
\newcommand{\Hom}[2]{\mathsf{Hom}(#1,#2)}
\newcommand{\eval}[0]{\mathsf{eval}}

\title{Difference in Complete Atomic Boolean Algebras}
\author{Prof. Harley Eades (heades@gru.edu).}
\date{\vspace{-22px}}
\begin{document}
\maketitle

\section{Complete Atomic Boolean Algebras}
\label{sec:complete_atomic_boolean_algebras}
\vspace{10px}
\begin{definition}
  \label{def:atomic}
  Suppose $(A, \leq, \perp)$ is a poset with a least element $\perp$.
  An element $a \in A$ is called \textbf{atomic} if for any $b \leq a$
  we have $b = \perp$ or $b = a$.
\end{definition}


\vspace{10px}
\begin{definition}
  \label{def:CABA}
  A \textbf{complete atomic boolean algebra (CABA)} is defined by the
  following axioms on a poset $(B, \leq)$:
  \begin{itemize}
  \item (Top) There is an element $\top \in B$ such that for any $x \in B$, $x \leq \top$.
  \item (Bottom) There is an element $\perp \in B$ such that for any $x \in B$, $\perp \leq x$.
  \item (Meet) Given elements $a,b \in B$, there is an element $a \times b \in B$, such that,
    $x \leq a$ and $x \leq b$ iff $x \leq a \times b$.
  \item (Join) Given elements $a,b \in B$, there is an element $a + b \in B$, such that,
    $a \leq x$ and $b \leq x$ iff $a + b \leq x$.
  \item (Complement) Given an element $a \in B$ there is an element $\lnot a \in B$, such that,
    $a \times \lnot a \leq \perp$ and $\top \leq a + \lnot a$.
  \item (Distributivity) Given elements $a, b, c \in B$ the following
    must hold $a \times (b + c) \leq (a \times b) + (a \times c)$.
    
  \item (Atomic) Given any element $b \in B$ we have that $b =
    \bigplus_{i \in I} a_i$ where each element of ${a_i}_{i \in I} \subseteq B$ is atomic.
    
  \item (Complete) Every $B' \subseteq B$ has a supremum (least-upper bound).
  \end{itemize}
\end{definition}
% section complete_atomic_boolean_algebras (end)

\section{CABAs are Powerset Algebras}
\label{sec:cabas_are_powerset_algebras}

Given a CABA $(B,+,\times,\lnot,\perp,\top)$ we can define a powerset
algebra $(\powerset{X},\cup,\cap,
\overline{\stackrel{\,\,\,\,}{\stackrel{\,\,\,\,}{\,\,\,\,}}},\emptyset,X)$.
Suppose $A \subseteq B$ is the set of atomic elements.  Then set $X =
A$.  So we will define the powerset algebra $P = (\powerset{A},\cup,\cap,
\overline{\stackrel{\,\,\,\,}{\stackrel{\,\,\,\,}{\,\,\,\,}}},\emptyset,A)$.

(\textbf{Atomicity}) Notice that we have an isomorphism between
$A$ and the set of single element subsets of $A$.  Thus, we define any
subset of $X \subseteq A$ as the union of single element subsets of
$A$.  This, implies that $P$ is atomic.

(\textbf{Completeness}) We can also show that $P$ is complete.  Given
any set $L \subseteq \powerset{A}$ the sup of $L$ is defined to be $L
\subseteq \bigcup L$.  This is clearly least among all sets containing
$L$.

(\textbf{Greatest Element}) The greatest element is defined to be $A$
because all subsets of $A$ are contained in $A$.

(\textbf{Least Element}) The least element is defined to be the empty
set, $\emptyset$, which is trivially contained in every set.

(\textbf{Meet}) Suppose $b_1 \times b_2 \in B$ then by atomicity we
know $b_1 = \bigplus_{i \in I} \{a_i\}$ and $b_1 = \bigplus_{j \in J}
\{a_j\}$ for atomic variables $a_i$ and $a_j$.  We map $\times$ to
$\cap$, and $b_1$ to the set $\{a_i\}$ and $b_2$ to the set $\{a_j\}$.
Thus, we have $\{a_i\} \cap \{a_j\} \in P$, and the meet condition is
clearly satisfied.

(\textbf{Join}) Mapping an element $b_1 + b_2 \in B$ is similar to the
previous case.

(\textbf{Complement}) Given an element $L \in \powerset{A}$ we define
$\comp{L} \in \powerset{A}$ as the complement relative to $A$. Thus,
$\comp{\emptyset} = A$ and $\comp{A} = \emptyset$.  Furthermore, for
$L \in P$, $L \cap \comp{L} = \emptyset$ and $L \cup \comp{L} = A$.
Finally, mapping any element $\lnot b \in B$ can be mapped to the
complement of the set of atomic elements defining $\lnot b$.

(\textbf{Distributivity}) It is an elementary fact of set theory that
intersection distributes over union.

These facts show that any CABA can be mapped to a powerset algebra. It
is also easy to see that every powerset algebra is a CABA. Thus, an
isomorphism.  
% section cabas_are_powerset_algebras (end)

\section{The Category of CABAs}
\label{sec:the_category_of_cabas}
We now define the category of CABAs by taking CABAs as objects and
CABA homomorphisms as morphisms.  We call this category
$\CABAC$.
\vspace{10px}
\begin{definition}
  \label{def:caba-morph}
  A \textbf{CABA homomorphism} between CABAs
  $(B_1,+,\times,\lnot,\perp,\top)$ and
  $(B_2,+,\times,\lnot,\perp,\top)$ is a function $H : B_1 \to B_2$
  such that the following holds:
  \begin{itemize}
  \item[1.] $H(\top) = \top$
  \item[2.] $H(\perp) = \perp$
  \item[3.] $H(b_1 \times b_2) = H(b_1) \times H(b_2)$
  \item[4.] $H(b_1 + b_2) = H(b_1) + H(b_2)$
  \item[5.] $H(\lnot b) = \lnot H(b)$
  \item[6.] $H$ maps sups of $B_1$ to sups of $B_2$
  \end{itemize}  
\end{definition}
It is easy to see that we can define identity CABA homomorphisms, and
composition of CABA homomorphisms is simply set theoretic
composition, which, is associative and respects identities.

\subsection{Coproducts in $\CABAC$}
\label{subsec:coproducts_in_cabac}
The coproduct construction on CABAs is not obvious at first, but by
working with powerset algebras we can more easily see how the
construction should be done.

Given two powerset algebras $(\powerset{X},\cup,\cap,
\overline{\stackrel{\,\,\,\,}{\stackrel{\,\,\,\,}{\,\,\,\,}}},\emptyset,X)$
and
$(\powerset{Y},\cup,\cap,
\overline{\stackrel{\,\,\,\,}{\stackrel{\,\,\,\,}{\,\,\,\,}}},\emptyset,Y)$
we can define their coproduct as the powerset algebra
$(\powerset{X \times Y},\cup,\cap,
\overline{\stackrel{\,\,\,\,}{\stackrel{\,\,\,\,}{\,\,\,\,}}},\emptyset,X \times Y)$.
Both injections can be defined as follows:
\begin{center}
  \begin{math}
    \begin{array}{lll}
      \begin{array}{lll}
      i_1 : \powerset{X} \to \powerset{X \times Y}\\
      i_1(S \subseteq X) = \{(x,y) \in X \times Y \mid x \in S\}\\
      \end{array}\\
      \\
      \begin{array}{lll}
        i_2 : \powerset{Y} \to \powerset{X \times Y}\\
        i_2(S \subseteq Y) = \{(x,y) \in X \times Y \mid y \in S\}\\
    \end{array}
    \end{array}
  \end{math}
\end{center}
It is easy to see that the previous injections are homomorphisms that
preserve the structure of the powerset algebra.

At this point we must make the following diagram commute:
\begin{center}
  \begin{math}
    \bfig
    \Atrianglepair/<-`<--`<-`->`<-/<900,700>[\powerset{Z}`\powerset{X}`\powerset{X \times Y}`\powerset{Y};f`h`g`i_1`i_2]
    \efig
  \end{math}
\end{center}
First, note that any subset $S \in \powerset{X \times Y}$ can be written as
$S = \inO{S_X} \cup \inT{S_Y}$ where $S_X \subseteq X$ and $S_Y \subseteq Y$. We define $h$ as follows:
\begin{center}
  \begin{math}
    \begin{array}{lll}
      h : \powerset{X \times Y} \to \powerset{Z}\\
      h(S \subseteq X \times Y) = f(\pi_1(S)) \cap g(\pi_2(S))\\
    \end{array}
  \end{math}
\end{center}
Given the previous definitions we can now see that the previous diagram commutes by construction.

The definition of coproduct for powersets gives us a hint of how to do
it for general CABAs, certainly, we know they must exist in $\CABAC$
because it is isomorphic to the category of powerset algebras.  The
coproduct construction just given shows that we must define the
coproduct in terms of the disjoint union of the sets of atomic
elements. In addition, notice that the least elements of both powerset
algebras in the coproduct are the same, and the greatest element of
the coproduct is defined in terms of the greatest elements of the two
component powerset algebras.  This implies that we must collapse the
greatest and lowest elements of two arbitrary CABAs in order to
construct the coproduct.  

Suppose $(B_1,+_1,\times_1,\lnot_1,\perp_1,\top_1)$ and
$(B_2,+_2,\times_2,\lnot_2,\perp_2,\top_2)$ are two CABAs.
Furthermore, suppose $A_1$ and $A_2$ are the sets of atomic elements
of $B_1$ and $B_2$ respectively.  Now take the set $A_1 \oplus A_2$ to
be the coequalizer:
\begin{center}
  \begin{math}
    1 \two^{[\inO{\perp_1},\inT{\top_1}]}_{[\inO{\perp_2},\inT{\top_2}]} A_1 + A_2 \to A_1 \oplus A_2
  \end{math}
\end{center}
Thus, the set $A_1 \oplus A_2$ is the set of equivalence classes where
the only identified elements are the least and greatest elements.
That is, $\inO{\perp_1} = \inT{\perp_2}$ and $\inO{\top_1} =
\inT{\top_2}$.  Every other equivalence class consists of a single
element of $A_1 + A_2$.  We denote each element, $[\inO{a_i}]$ and
$[\inT{a_j}]$, by simply $a_i$ and $a_j$, and then denote the
collapsed least and greatest elements by $\perp_\oplus$ and
$\top_\oplus$.  We can now define the preorder on $A_1 \oplus A_2$,
denoted by $\leq_\oplus$, as follows:
\begin{center}
  \begin{math}
    \begin{array}{lll}
      a_1 \leq_\oplus a_2 & \text{iff } a_1 \leq_1 a_2 \text{ or } a_1 \leq_2 a_2\\
      \perp_\oplus \leq_\oplus a & \text{iff } a \in A_1 \text{ or } a \in A_2\\
      a \leq_\oplus \top_\oplus & \text{iff } a \in A_1 \text{ or } a \in A_2\\
    \end{array}
  \end{math}
\end{center}
% subsection coproducts_in_$\cabac$ (end)

\subsection{Cointernal Hom in $\CABAC$}
\label{subsec:cointernal_hom_in_CABACat}
The internal hom in $\Set$ is defined by $A \imp B = \{f \mid f : A
\to B \text{ is a set function} \}$.  Curry can be defined as follows:
\[
\begin{array}{lll}
  \cur : \Hom{A \times B}{C} \to \Hom{A}{B \imp C}\\
  \cur(f : A \times B \to C) = \lambda a.\lambda b.f(a,b).  
\end{array}
\]
It is easy to see that $\cur$ is indeed an isomorphism.  The evaluator
can be defined as follows:
\[
\begin{array}{lll}
  \eval : (A \imp B) \times A \to B\\
  \eval = \cur^{-1}(\id_{A \imp B})
\end{array}
\]

The following contravariant functor is called the \textbf{powerset functor}:
\[
\begin{array}{lll}
  P : \Set \to \Set^{op}\\
  P(X) = \powerset{X}\\
  P(f : A \to B)(B' \subset B) = f^{-1}(B') = \{a \in A \mid \exists b \in B'.f(a) = b \}\\
\end{array}
\]
Now we can compute the opposite of the internal hom:
\[
\begin{array}{lll}
  P(A \imp B) = \powerset{A \imp B}\\
\end{array}
\]

\[
\cocur : \Hom{\powerset{B \imp C}}{\powerset{A}} \to \Hom{\powerset{C}}{\powerset{A \times B}}
\]
% subsection cointernal_hom_in_cabacat (end)

% section the_category_of_cabas (end)


\end{document}
