TODO

The most comprehensive treatment of ILL is in Gavin Bierman's thesis
\cite{Bierman:1994}.  There one finds the Proof Theory (Chapter 2),
i..e, the sequent calculus with cut-eliminaton, natural deduction and
axiomatic versions of ILL. Then (Chapter 3) a term assignment to the
natural deduction and to the sequent calculus versions are presented
with $\beta$-reductions and commutative conversions, and strong
normalization and confluence are proved for the resulting calculus. A
painstaking analysis of the rules of the labelled calculus allows us
to construct a categorical model of ILL, in particular of the
exponential part, which is a main contribution of Bierman and of the
Cambridge school of the 1990s with respect to previous models by Seely
and Lafont.

Benton's work \cite{Benton:1994} on LNL logic, which we read in a TeX
report, is much less systematc and detailed than Bierman's, but
presents effectively the categorical model of LNL (Chapter 2) shows
how to obtain a LNL model from a Linear Category and viceversa. Then
versions of the sequent calculus for LNL are considered and
cut-elimination is proved for one calculus. Then Natural Deduction is
given with term assignment and the categorical interpretaiton of a
fragment of the natural deduction system. $\beta$ reductions and
commuting conversions are then presented.  Undoubtedly N.Benton's TeX
report provides an agile tool which we followed very closed in our
investigation.
