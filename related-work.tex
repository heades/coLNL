The most comprehensive treatment of ILL is in Gavin Bierman's thesis
\cite{Bierman:1994}.  There one finds the Proof Theory (Chapter 2),
i.e, the sequent calculus with cut-eliminaton, natural deduction and
axiomatic versions of ILL. Then (Chapter 3) a term assignment to the
natural deduction and to the sequent calculus versions are presented
with $\beta$-reductions and commutative conversions, and strong
normalization and confluence are proved for the resulting calculus. A
painstaking analysis of the rules of the labeled calculus leads to the
construction of a categorical model of ILL, a \emph{linear category},
in particular of the exponential part, a main contribution of Bierman
and of the Cambridge school of the 1990s with respect to previous
models by Seely and Lafont.  Bellin \cite{Bellin:2012} presents a
categorical model of co-intuitionistic linear logic based on a
dualization of Bierman \cite{Bierman:1994} construction for ILL.

Benton's work \cite{Benton:1994} on LNL logic presents the categorical
model for Linear-Non-Linear Intuitionistic logic LNL.  Chapter 2 shows
how to obtain a LNL model from a Linear Category and viceversa.
Versions of the sequent calculus for LNL are considered and
cut-elimination is proved for one such version. Then Natural Deduction
is given with term assignment and the categorical interpretation of a
fragment of the natural deduction system. Then $\beta$-reductions and
commuting conversions are presented.  The present work follows
Benton's paper aiming at a (non-trivial) dualization of it.

Bi-intuitionistic logic was introduced by C.Rauszer
\cite{Rauszer:1974} with an algebraic and Kripke semantics
\cite{Rauszer:1980} and a Gentzen style sequent calculus
\cite{Rauszer:1974a}.  Co-intuitionistic logic requires a multiple
conclusion system, because of the cotensor in the linear case and of
contraction right in the non-linear one.  This raises the problem of
the relations between intuitionistic implication and disjunction, and,
dually, between subtraction and conjunction.  In the case of the logic
FILL that extends ILL with the cotensor (\emph{par}) applying Maheara
and Dragalin's restriction that only one formula occurs in the
succedent of the premise of an implication right, yields a calculus
that does not satisfies cut-elimination, as noticed by Schellinx
\cite{Schellinx:1991}. Similarly, in the logic BILL
(\emph{Bi-Intuitionistic Linear Logic}) requiring that only one
formula occurs in the antecedent of the premise of a subtraction left
yields a system that does not satisfy cut-elimination.

\begin{center}
\begin{tabular}{ccc}
\AxiomC{$\Gamma, A \vdash B$}
\RightLabel{$\limp$ R}
\UnaryInfC{$\Gamma \vdash A \limp B$}
\DisplayProof & \hskip1in\strut& 
\AxiomC{$ A \vdash B, \Delta$}
\RightLabel{$\lsub$ E}
\UnaryInfC{$A \lsub B \vdash \Delta$}
\DisplayProof 
\end{tabular}
\end{center}
As a simple counterexample, consider the sequent $p \Rightarrow q, r
\rightarrow ((p - q) \wedge r)$ given by Pinto and Uusatlu around 2003
\cite{Pinto-Uustalu:2010}, which is provable with cut but not cut-free
with Dragalin's restrictions.

Hyland and de Paiva introduced a sequent calculus for FILL labeled
with terms
\begin{center}
\begin{tabular}{c}
\AxiomC{$\overline{y}:\Gamma, x:A \vdash t:B, \overline{u}:\Delta$}
\RightLabel{$\limp$ R}
\UnaryInfC{$\overline{y}: \Gamma \vdash \lambda x:T A \limp B, \overline{u}:\Delta$}
\DisplayProof
\end{tabular} 
\end{center}
where $x: A$ occurs in $t:B$ if and only if there is an ``essential
dependency'' of $B$ from $A$.  The restriction on the $\limp$ I is
that $x$ does not occur in the terms $\overline{u}:\Delta$.  The
original term assignment did not guarantee cut-elimination, as noticed
by Bierman \cite{Bierman:1996}; the assignment to \emph{par left}
($\oplus$ L) had to be fine tuned, as indicated by Bellin
\cite{Bellin:1997}.

%% \footnote{For \emph{par} left ($\oplus L$) we need the following, :
%% where $x: A$ occurs in $t:B$ if and only if there is an ``essential
%% dependency'' of $B$ from $A$.  The restriction on the $\limp$ I is
%% that $x$ does not occur in the terms $\overline{u}:\Delta$.  The
%% original term assignment did no guarantee cut-elimination, as noticed
%% by Bierman \cite{Bierman:1996}; the assignment to \emph{par left}
%% ($\oplus$ L) had to be fine tuned, as indicated by Bellin
%% \cite{Bellin:1997}
%% \footnote{For \emph{par} left ($\oplus L$) we need the following:
%% \[
%% \AxiomC{$x:A, \overline{x}:\Gamma\vdash \overline{r}: \Delta\qquad y:B, \overline{y}:\Pi\vdash \overline{s}: \Lambda$}
%% \UnaryInfC{$z: A \oplus B, \overline{x}:\Gamma, \overline{y}:\Pi\vdash  \overline{r'}: \Delta, 
%% \overline{s'}: \Lambda $}
%% \DisplayProof
%% \]
%% where for $r'_i\in \overline{r'}$ and $s'_j \in \overline{s'}$ we have 
%% \[
%% \begin{tabular}{rll}
%% $r'_i =$ & $\mathtt{let}\ z\ \mathtt{be}\ x-\ \mathtt{in}\ r_i, \qquad$ & if $x$ occurs in $r_i$, \\
%%              & $r_i$                                                                              & otherwise.\\
%% $s'_j =$ & $\mathtt{let}\ z\ \mathtt{be}\ -y\ \mathtt{in}\ s_j, \qquad$ & if $y$ occurs in $s_j$, \\
%%              & $s_j$                                                                              & otherwise.\\
%% \end{tabular}
%% \]
%% We may introduce non-existent dependencies if we define always $r'_i =
%% \mathtt{let}\ z\ \mathtt{be}\ x-\ \mathtt{in}\ r_i.$
%% }

A detailed presentation of the term calculus for FILL with a full
proof of cut elimination by Eades and de Paiva is in
\cite{EadesP:2016}, where the correctness for a categorical semantics
for FILL is also proved.  Another correct formalization of FILL, a
sequent calculus with a relational annotation, was given by Bra\"uner
and de Paiva \cite{BraunerDePaiva:1997}, with a proof of
cut-elimination.  The second author \cite{Bellin:1997} gave a system
of proof nets for FILL which sequentialize in the sequent calculus
with term assignment; the essential fact here is that \emph{$x:A$
  occurs in $t:B$ if and only if there is a ``directed chain'' between
  $A$ and $B$ in the proof structure.}  Here cut elimination is proved
by reduction to cut-elimination for proof nets.


A system of two-sided proof nets (in the style of natural deduction)
was given by Cockett and Seely \cite{Cockett:1997}.  For
Bi-Intuitionistic Linear Logic, they gave also a system of proof nets,
corresponding to a sequent calculus without annotations and
restrictions that therefore collapses into classical MLL.  Recently,
Clouston, Dawson, Goré and Tiu \cite{CloustonDGT:2013} gave an 
annotation-free formalization for BILL, alternative to sequent calculi, 
in the form of deep-inference and display calculi for BILL. This calculus 
enjoys cut-elimination and is relevant to the categorical semantics
bi-intuitionistic linear logic. Annotation-free formalizations of 
Bi-Intuitionistic Logic use the display calculus \cite{Gore:2000}, nested sequents
\cite{GorePT:2008} and deep inference \cite{Postniece:2009}.
 

Tristan Crolard \cite{Crolard:2001,Crolard:2004} made an in-depth
study of Rauszer's logic. In \cite{Crolard:2001} he showed that models
of Rauszer logic (called ``subtractive logic'') based on bi-cartesian
closed categories (with co-exponents) collapse to preorders.  He also
studied models of subtractive logic and showed that its first order
theory is constant-domain logic, thus it is not a conservative
extension of intuitionistic logic.

Crolard \cite{Crolard:2004} develops the type theory for subtractive
logic, extending a system of multiple conclusion classical natural
deduction with a connective of subtraction and then decorating proofs
with a system of annotations of dependencies that allows us to
identify ``constructive proofs'': these are derivations where only the
premise of an implication introduction depends on the discharged
assumption and only the premise of a subtraction elimination depends
on the discharged conclusion. Therefore Crolard's sequent calculus
with annotations is not affected by the counterexamples to
cut-eliminations.

The type theory is Parigot $\lambda\mu$-calculus extended with
operators for sums, products and subtraction, where the operators for
subtraction introduction and elimination are understood as a calculus
of co-routines.  A constructive system of co-routines is then obtained
by imposing restrictions on terms corresponding to the restrictions on
constructive proofs.  

In a series of papers the second author gave a ``pragmatic''
interpretation of bi-intuitionism, where intuitionistic and
co-intuitionistic logic are interpreted as logics of the acts of
assertion and making a hypothesis, respectively, the interactions
between the two sides depending on negations, see \cite{Bellin:2014}.
Here the separation between intuitionistic and co-intuitionistic logic
and their models is given a linguistic motivation. Writing $\vdash p$
for the type of assertions that $p$ is true and using intuitionistic
connectives with the BHK interpretation, one gives a ``pragmatic
interpretatiion'' of ILL, where an expression $A$ is \emph{justified}
or \emph{unjustified} \cite{dalla1995pragmatic}. Similarly, writing
$\mathcal{H}\; p$ for the type of hypotheses that $p$ is true, and
using co-intuitionistic connectives, one builds a co-intuitionistic
language, for which an analogue ``pragmatic interpretation'' has been
attempted.  Both languages may be given a modal interpretattion in S4,
with $(\vdash p)^M = \Box p$ and $(\mathcal{H}\; p)^M = \diamondsuit
p$. Notice that here there is a semantic duality between an assertion
$\vdash p$ and a hypothesis $\mathcal{H}\; \neg p$, as $\Box p$ and
$\diamondsuit \neg p$ are contradictory. Similarly there is a semantic
duality between $\mathcal{H}\; p$ and $\vdash \neg p$, but not between
$\vdash p$ and the hypothesis $\mathcal{H}\, p$. A useful direction of
research in the proof theory of bi-intuitionism may be the
investigation the relations between co-intuitionistic proofs and
intuitionistic refutations.

It is in this context that a term assignment for co-intuitionistic  logic 
has been developed, starting from Crolard's definition but independently
of the $\lambda\mu$-framework. This calculus was used
here as a term assignment of Dual LNL logic.

Trafford \cite{trafford2016structuring} defines an interpretation of
co-intuitionistic logic into a topos-theoretic model to represent both
proofs, in an elementary topoi, and refutations, in a complement
topoi.  He then shows that classical logic can be simulated in his
model.  Earlier Estrada-Gonz\'alez \cite{estrada2010complement} gave a
sequent calculus for BINT based on complement topoi.

Finally, to achieve the project outlined in the introduction of putting
together intuitionistic and co-intuitionistic adjoint logic in the
environment of BILL the definition of a suitable syntax for BILL will
play a key role.

%% %%% Local Variables: 
%% %%% mode: latex
%% %%% TeX-master: main.tex
%% %%% End: 
